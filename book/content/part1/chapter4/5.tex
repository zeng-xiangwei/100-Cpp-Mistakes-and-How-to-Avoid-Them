此错误会影响可读性和有效性。文件系统在不同平台上有所不同,开发与它们交互的代码可能很复杂、晦涩难懂且不可移植。

\mySamllsection{问题}

许多程序必须访问文件系统才能输入和输出数据,以进行配置、处理 、记录结果或其他目的。多年来,实现此目的所需的代码在复杂性和功能性方面不断增长。这一进程的一个重大问题是,不同的平台已经朝着不同的、不兼容的方向发展。

编写一个需要检查文件是否存在并在不存在时创建文件的程序很简单,但要考虑到平台文件系统的复杂性。开发人员需要编写一个简单的程序来检查文件是否存在;如有必要,创建文件;并输出一行文本。
该程序必须可在 Windows 和 Linux 上运行。虽然为任一系统编写功能都很简单,但为两者编写功能却很复杂。有两种方法:预处理器指令和单独的源文件。第一种方法没有演示,但多个代码库使用了这种技术。清单 4.10 和 4.11 中的代码演示了使用单独文件的第二种方法。W indows 文件处理依赖于几个特定于平台的函数和常量。

\filename{清单4.10 创建和填充文件:Windows}

\begin{cpp}
#include <windows.h> // 1

void createFileIfNotExists(const std::string& filename) {
  std::wstring wideFilename = std::wstring(filename.begin(), filename.end());
  if (GetFileAttributesW(wideFilename.c_str()) == INVALID_FILE_ATTRIBUTES) {
    HANDLE hFile = CreateFileW(wideFilename.c_str(), GENERIC_WRITE, 0,
    NULL, CREATE_NEW,
    FILE_ATTRIBUTE_NORMAL, NULL); // 2
    if (hFile != INVALID_HANDLE_VALUE) { // 3
      std::wcout << L"File created: " << wideFilename << std::endl;
      const wchar_t* content = L"Hello, world!\r\n";
      DWORD bytesWritten;
      WriteFile(hFile, content, wcslen(content) * sizeof(wchar_t),
      &bytesWritten, NULL);
      CloseHandle(hFile);
    } else
      std::cerr << "Error creating file: " << filename << std::endl; // 4
    } else
      std::wcout << L"File already exists: " << wideFilename << std::endl;
}

int main() {
  createFileIfNotExists("example_file.txt");
  return 0;
}
\end{cpp}

{\footnotesize
注释1:特定于windows的包含文件

注释2:windows特有的常量

注释3:针对windows的使用

注释4:使用宽字符流时使用wcerr
}

Linux 拥有一些特定于平台的函数和常量,它们与 Windows 中使用的不同。这种差异使得跨平台处理变得困难且容易出错。专门从事某一平台的程序员会对另一平台开发人员的智慧发出尖刻的咒骂。

\filename{清单4.11 创建和填充文件:Linux}

\begin{cpp}
#include <unistd.h> // 1
#include <fcntl.h> // 1

void createFileIfNotExists(const std::string& filename) {
  if (access(filename.c_str(), F_OK) == -1) {
    int fd = open(filename.c_str(), O_CREAT | O_WRONLY, S_IRUSR
         | S_IWUSR | S_IRGRP |S_IROTH); // 2
    if (fd != -1) { // 3
      std::cout << "File created: " << filename << std::endl;
      const char* content = "Hello, world!\n";
      write(fd, content, strlen(content));
      close(fd);
    } else
      std::cerr << "Error creating file: " << filename << std::endl;
    } else
      std::cout << "File already exists: " << filename << std::endl;
}

int main() {
  createFileIfNotExists("example_file.txt");
  return 0;
}
\end{cpp}

{\footnotesize
注释1:Linux 特定的包含文件

注释2:Linux 特定的常量

注释3:Linux 特定的用法
}

\mySamllsection{分析}

一种常见的方法是使用预处理器指令,这些指令可以根据平台选择性地包含或排除代码。所有代码都包含在一个源文件中,任何变体都按平台分隔,并通过 \#ifdef 指令启用。这样做的好处是简化了构建过程。单个源文件易于编写和构建;但是,使用可选包含来复制代码通常更难阅读。

这里演示的第二种方法是使用单独的文件,并通过一些特定于平台的宏或其他技术确定哪些文件包含在构建过程中。将一个平台的所有代码放在一个源文件中,这样做的简单性使阅读更容易,但使构建变得复杂。

主要问题是跨平台代码很少用于特定于平台的操作,例如文件系统操作。同样,我们的开发人员必须花费大量时间来处理那些不直接影响正在解决的问题但属于内部细节的错误,这些错误虽然必不可少,但通常很难管理。

\mySamllsection{解决}

现代 C++17 提供了一组标准函数,适用于任何符合要求的平台上的文件系统。通用性允许直接编写以通用方式执行文件系统功能的代码。清单 4.12 中的代码将前面的代码示例重新编写为一种适用于 Windows 和 Linux 的通用形式。此解决方案无需使用预处理器指令或多个源文件进行条件编译来隔离特定于平台的代码。执行了几个错误检查测试,因为文件系统在压力下表现不佳(但在开发过程中很少出现)。

\filename{清单4.12 创建和填充文件:通用}

\begin{cpp}
namespace fs = std::filesystem; // 1

void createFileIfNotExists(const
    std::string& filename) { // 2
  fs::path filePath = filename;
  if (fs::exists(filePath)) { // 3
    std::cout << "File already exists: " << filePath << std::endl;
    return;
  }
  std::ofstream file(filePath);
  if (!file) {
    std::cerr << "Error creating file: " << filePath << std::endl;
    return;
  }
  if (!file.is_open()) {
    std::cerr << "File cannot be opened: " << filePath << std::endl;
    return;
  }
  std::cout << "File created: " << filePath << std::endl;
  file << "Hello, world!\n"; // 4
}

int main() {
  createFileIfNotExists("example_file.txt");
  return 0;
}
\end{cpp}

{\footnotesize
注释1:为更简单的使用而对命名空间进行别名的便捷方法;代码中的 fs 别名为 std::filesystem

注释2:适用于 Windows 和Linux 的通用代码

注释3:检查时间

注释4:使用时间
}

将文件系统功能概括为标准标头,使开发人员能够专注于使用文件系统原语解决问题。文件系统的简化抽象以与任何特定系统无关的方式提供了所有基本功能。该指令指定一个别名;此形式可用于多种上下文(例如模板)。在这里,它简化了在 path 变量和 exists 函数之前写入 std::filesystem 的过程;只需使用 fs 别名和范围解析运算符 (::) 即可指定它们。太棒了!清单 4.12 中的代码有一个重要的警告:该示例仅适用于单线程环境。

并发增加了复杂性,代码中没有演示这一点。问题是另一个线程可能在检查时间 (TOC) 和使用时间 (TOU) 之间删除、更新或以其他方式影响文件的特性。自然,这个问题被称为 TOCTOU;它代表了一种竞争条件,可能会让粗心的开发人员犯错。

\mySamllsection{建议}

\begin{itemize}
\item
使用新的文件系统功能来处理文件和目录。

\item
尽可能更新现有代码以使用新功能。

\item
访问文件系统后记得检查错误;虽然代码更 容易编写,但错误仍然会带来问题。
\end{itemize}














