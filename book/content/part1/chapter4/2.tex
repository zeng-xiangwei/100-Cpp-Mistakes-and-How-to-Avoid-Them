这种错误会影响有效性和可读性,并且会对正确性产生重大的影响。枚举允许开发人员命名以增强可读性,但必须正确且清晰。

\mySamllsection{问题}

变量和常量可用于命名概念,例如:可以为整数常量分配一些映射到红色或蓝色的值。除非开发人员小心谨慎,否则这种映射可能会非常随意,阅读时可能会很难懂。C++ 语言提供了一种简化方法,即使用 enum 常量。枚举中的命名对象可以表示集合中的可能值。

枚举常量以整数形式实现,因此很容易与整数相互转换。清单 4.3 显示了一个表示颜色的整数变量,该变量为 2,这是什么意思呢?

开发人员正在解决一个与汽车及其相互作用有关的问题。汽车和交通信号灯都有颜色。枚举颜色以便使用符号名称是有意义的。编译器默认为这些常量分配值,并且会自动考虑更改(即添加、删除或重新排列)。然而,开发人员遇到了一些问题。以下代码显示了对汽车和交通信号灯的颜色属性进行建模(简陋)的尝试。

\filename{清单4.3 有问题的全局命名空间枚举}

\begin{cpp}
enum TrafficColor {
  Green,
  Yellow, // 1
  Red // 2
};

enum PaintColor {
  Gray,
  White, // 2
  Black,
  Paint_Red, // 3
  Blue,
  Paint_Yellow,
  Paint_Green
};

int main() {
  int color = 2; // 4
  switch (color) {
  case Red:
    std::cout << "we have a red one\n";
    break;
  case Blue:
    std::cout << "we have a blue one\n";
    break;
  case Yellow:
    std::cout << "we have a yellow one\n";
    break;
  default:
    std::cout << "we have a white one\n";
    break;
  // case White: // 5
  };
  return 0;
}
\end{cpp}

{\footnotesize
注释1:红色交通信号灯进入全局命名空间

注释2:两个枚举中的索引默认为1

注释3:不能重复使用完全相同的枚举名称;必须对其进行修改

注释4:为枚举分配和使用 int

注释5:枚举中的重复索引值——无法定义
}

\mySamllsection{分析}

枚举常量位于全局命名空间中。如果声明的符号不包含在特定命名空间中,则该命名空间是每个符号所在的位置。开发人员在使用enum 常量时发现了几个问题,如清单 4.3 所示。

首先,所有常量都在一个命名空间中,因此重复值会造成错误。重命名油漆颜色的努力表明开发人员,必须弱化符号名称才能解决这个问题。

其次,由于整数支持常量,任何混合两个 enum 集的代码都存在多个冲突值的风险,会产生歧义或错误。 switch 语句不能有两个枚举常量值相同的情况,例如 Yellow 对应 TrafficColor 和White 对应 PaintColor(每个都分配了相同的值;在本例中为 1) 。

第三,可以给已分配颜色的变量分配任意整数值。此示例为颜色分配了 2。开发人员必须努力清楚地传达分配的含义。需要在示例中明确说明,值为 2 的枚举常量指的是哪个。

最后,支持整数变量无法知道某个值是否超出范围。该变量可能赋值为负数或大于最大有效常数值,很难搞定清楚这代表了什么。更糟糕的是,代码可能会继续运行,但会意外或错误地处理不正确的赋值(更多未定义行为)。

\mySamllsection{解决}

现代 C++ 来了!青天就有了!枚举现在可以封装在类命名空间中,这让全局命名空间井然有序,允许在不同命名空间中使用相同的常量名称。尽管支持变量类型是整数,但代码无法为该变量分配整数值,分配必须在该类的枚举值集内,不能分配非法值。

以下代码显示了如何使用 enum 类,将所有定义的值隔离到一个类中,而不会与另一个类冲突。一个类(和命名空间)的常量不能分配给另一个变量,从而避免了前面遇到的一些问题。

\filename{清单4.4 单独的命名空间枚举}

\begin{cpp}
enum class Traffic { // 1
  Green,
  Yellow,
  Red
};

enum class Paint { // 2
  Gray,
  Black,
  White,
  Red,
  Blue,
  Yellow,
  Green
};

int main() {
  auto color = Traffic::Yellow; // 3
  switch (color) {
  case Traffic::Red: // 4
    std::cout << "we have a red light\n";
    break;
  case Traffic::Yellow:
    std::cout << "we have a yellow light\n";
    break;
  case Traffic::Green:
  default:
    std::cout << "we have a green light\n";
    break;
  };

  Paint can = Paint::Blue;
  switch (can) {
  case Paint::Red:
    std::cout << "we have a red cube\n";
    break;
  case Paint::Yellow:
    std::cout << "we have a blue cube\n";
    break;
  case Paint::White:
    [[fallthrough]];
  default: // 5
    std::cout << "we have a white cube\n";
    break;
  };
  return 0;
}
\end{cpp}

{\footnotesize
注释1:命名空间绑定枚举

注释2:不同命名空间绑定枚举

注释3:使用 auto 作为数据类型,该数据类型由初始化值推导而来

注释4:与同名其他类中的枚举字面量不同

注释5:消除故意“贯穿”行为的警告
}

[[fallthrough]] 注释记录了不使用 break 关键字的有意失败行为的情况。 switch 语句很容易出错,因此请使用可用的技术来避免错误。不要忘记最后一个右括号 (\}) 后面的语句结尾分号 (;)。

枚举类将值隔离到一个命名空间中,并防止将任意值分配给类变量。使用这一功能,enum 的经典问题即得到解决并消除了歧义。如果需要指定 enum 类的底层表示类型,可以通过基本类型扩展来完成:

\begin{cpp}
enum class Traffic : uint8_t {
  ...
};
\end{cpp}

\mySamllsection{建议}

\begin{itemize}
\item
将全局 enum 定义更改为 enum 类。
\end{itemize}
