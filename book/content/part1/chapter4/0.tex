本章内容

\begin{itemize}
\item
文本格式的改进

\item
正则表达式

\item
Lambda 表达式

\item
可变参数模板

\item
可移植文件系统代码
\end{itemize}

现代 C++ 扩展了可供开发人员使用的功能范围和深度。许多领域都得到了改进,精心设计和清晰的建议已纳入标准。本章只是对一些功能的简单介绍,这些功能使日常编程更简单、更正确、更易读。

几乎所有 C++ 开发人员都会从这些附加主题中受益,但有人认为一些关键方面被忽略了。这种评估是正确的,因为有太多漂亮的功能无法一一列举。但是,请将这项工作视为一种通用方法的起点,以获得一些方便的改进。

文本处理已得到改进,使用尚未实现的格式标头。通过编写代码来解析文本非常复杂且容易出错,但正则表达式功能可以简化开发人员的工作。

出于对函数式编程概念的关注,标准委员会采用了范围,并更好地与现场定义的函数(称为 lambdas)进行交互。关于 lambda 的书写量 非常大,但只要稍微了解一下它们的用法,就有望激发您学习和使用它们的兴趣。范围实现了 Linux 的过滤器概念(通常与管道符号一起使用:|),其中一个的输出成为下一个的输入,这些输出被组合成非常强大的管道。

文件系统代码传统上是为特定平台开发的。对于跨平台代码,使用某种条件编译机制来选择正确的平台相关代码。现代 C++ 对这些区别进行了抽象,允许单个代码单元在任何平台上工作。最后,通过定义数学常数(例如 pi\footnote{译者注:pi表示数学中的π(圆周率)})的新功能,使用千位分隔符可以更快地理 解多位文字数字,以及用户定义的文字值(可提高可读性并有助于单位转换),可读性得到了极大增强。











