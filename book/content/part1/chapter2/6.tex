这种错误会影响效率和可读性,并且会对性能产生负面影响。使用循环对容器进行索引是一种常见的习惯做法,但这种传统方法过于复杂。

\mySamllsection{问题}

开发人员经常必须遍历容器才能访问每个元素,这是通过创建一个循环控制变量来实现的,该变量从第一个索引开始,并在每次循环中单调递增(+1)。最后一个索引始终是容器的长度减一。清单 2.11 演示了一个通用的 sum 函数,它接受一个值vector,并返回它们的总和。我们假设这些是测试分数,然后计算平均值。循环没有什么特别之处;它是索引容器的通用解决方案。还有什么比这更简单的呢?

\filename{清单2.11 使用索引值遍历容器}

\begin{cpp}
template <typename T>
T sum(const std::vector<T>& values) {
  T sum = (T)0;
  for (int i = 0; i < values.size(); ++i) // 1
    sum += values[i];
  return sum;
}

int main() {
  std::vector<int> values;
  for (int i = 0; i < 10; ++i)
    values.push_back(10*i + i); // 1
  std::cout << sum(values) << '\n';
  return 0;
}
\end{cpp}

{\footnotesize
注释1:传统基于索引的方法
}

\mySamllsection{分析}

清单 2.11 中的代码可以工作,并且反映了当前代码库中的大多数循环。对于其传统实现,需要编写一些代码才能实现,并且专注于实现循环机制。即使一个人非常熟悉解决方案,读起来也很会尴尬。且需要输入的代码字符越多,引入错误的几率就会变大。此外,对于那些不熟悉这种习惯用法的人来说,可能会将延续条件改为

\begin{cpp}
i <= values.size()
\end{cpp}

我们知道这是错误的。直觉上感觉正确,但实际上却是一个差1错误。

\mySamllsection{解决}

现代 C++ 提供了基于范围的 for 形式来简化这种模式。基于范围的 for 删除了延续条件和更新部分,并修改了初始化部分。将冒号读作“in”\footnote{译者注:类似于Python语言。},例如“对于值中的(每个)元素,执行某事”,其中“某事”是循环体。

循环的机制很简单。声明一个变量,该变量按顺序分配每个容器元素的值。值就像索引方法一样,从索引0处的元素开始赋值,然后是索引1,依此类推,直到最后一个元素。循环的每次迭代都会将元素复制到变量中。可以在循环中修改变量,但由于是副本,相应的修改不会影响容器中的元素值,可将此循环视为只读语义。如果需要修改元素值,请对变量使用引用类型。

若元素类型是类,则每次都会调用复制构造函数——这可能比必要的代价高得多。最好将变量设为元素类型的引用。这时只复制数据的位置,从而最大限度地减少数据量。除非循环会修改元素,否则请使用 const 关键字。对于数字类型,如清单 2.12 所示,引用可能过多。当元素类型是类时,这种方法性能会很好。某些情况下可以使用指针变量,但引用形式消除了对指针语法的需要,从而提高了可读性。

\filename{清单2.12 使用基于范围的for循环遍历容器}

\begin{cpp}
template <typename T>
T sum(const std::vector<T>& values) {
  T sum {}; // 1
  for (const T& value : values) // 2
    sum += value;
  return sum;
}

int main() {
  std::vector<int> values;
  for (int i = 0; i < 10; ++i) // 3
    values.push_back(10*i + i);
  std::cout << sum(values) << '\n';
  return 0;
}
\end{cpp}

{\footnotesize
注释1:使用初始化程序获取类型的正确零形式

注释2:基于范围的 for 循环使用简单

注释3:基于索引的 for 循环感觉笨重
}

\mySamllsection{建议}

\begin{itemize}
\item
使用基于范围的 for 语句,可以消除传统基于索引循环的大量编码工作。

\item
元素值会复制到变量中,循环本质上为只读。

\item
如果元素复制成本高昂,请使用引用变量,最大程度减少数据移动。
\end{itemize}



















