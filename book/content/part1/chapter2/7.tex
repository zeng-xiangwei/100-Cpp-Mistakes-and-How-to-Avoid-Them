这个错误与可读性有关,并且在一定程度上与有效性有关。NULL 宏继承自 C,而在前现代 C++ 中最好使用用nullptr初始化指针。

\mySamllsection{问题}

清单 2.13 展示了一个 Student 类的简单模型。其中一个实例变量初始化为0。此外,由于代码需要指针,其也初始化为0。

\filename{清单2.13 0值的混合使用}

\begin{cpp}
struct Student {
  std::string name;
  int id;
  double gpa;
  Student(const std::string& name, int id, double gpa) : name(name),
          id(id), gpa(0) {
    if (gpa < 0)
      throw std::invalid_argument("gpa is negative");
  }
};

int main() {
  Student* sammy = 0; // 1
  Student* ginny = 0;
  Student* gene = 0;
  sammy = new Student("Samuel", 0, 3.75);
  ginny = new Student("Virginia", 1, 3.8);
  gene = new Student("Eugene", 2, 0);
  return 0;
}
\end{cpp}

{\footnotesize
注释1:比 NULL 宏好点,但没好多少
}

\mySamllsection{分析}

经典 C++ 需要表示不代表任何对象的指针,所以将指针初始化为0是个好的方法。但是,0不是指针,其可能会导致混乱。虽然许多开发人员都熟悉这种用法,而作为指针来说,0比 NULL 宏更难阅读。因为编译器足够聪明,可以获取整数文字值并正确使用指针。

\mySamllsection{解决}

现代 C++ 提供了 nullptr,即表示空值,且专用于指针。其优点是意图清晰,易于理解。

其另一个明显优点是,编译器不会将其与类型为 int 的函数参数混淆。

\filename{清单2.14 使用 nullptr 明确指针状态}

\begin{cpp}
struct Student {
  std::string name;
  int id;
  double gpa;
  Student(const std::string& name, int id, double gpa) : name(name),
          id(id), gpa(0) {
    if (gpa < 0)
      throw std::invalid_argument("gpa is negative");
  }
};

int main() {
  Student* sammy = nullptr; // 1
  Student* ginny = nullptr;
  Student* gene = nullptr;
  sammy = new Student("Samuel", 0, 3.75);
  ginny = new Student("Virginia", 1, 3.8);
  gene = new Student("Eugene", 2, 0);
  return 0;
}
\end{cpp}

{\footnotesize
注释1:比 NULL 宏好,比0值明确。
}

\mySamllsection{建议}

\begin{itemize}
\item
使用 nullptr 初始化未指向有效对象的指针。

\item
删除分配内存后,最好将相应指针赋值为nullptr,以避免出现使用错误。
\end{itemize}
