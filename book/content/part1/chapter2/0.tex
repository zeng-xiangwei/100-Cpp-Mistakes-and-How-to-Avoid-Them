本章内容

\begin{itemize}
\item
移动语义

\item
提供的类成员

\item
基于范围的循环

\item
智能指针
\end{itemize}

随着 C++11 标准的出现,我们增加了一些更改来增强语言功能、改进标准模板库 (STL)、提高性能以及简化语法和表达能力。虽然本书没有介绍,但增加了一些针对并发性的重大改进,包括线程和任务、错误检测、按时间顺序和日历增强功能以及编译时计算。

C++11 中引入的语言功能提供了巨大的优势。auto 关键字可实现类型推断、简化代码并提高可读性和有效性。基于范围的 for 循环简化了集合上的迭代,提高了代码清晰度并减少了潜在错误。nullptr 文字通过区分空指针和整数零来提高安全性。

C++11 中引入的 STL 增强功能提供了显著的好处。智能指针的添加有助于安全内存管理并减少内存泄漏。移动语义简化了资源处理,通过减少不必要的复制来提高性能。新的容器丰富了数据结构选项,提高了数据处理的灵活性和效率。Lambda 表达式允许简洁而富有表现力的代码,从而提高可读性和有效性。

引入的性能优势提高了程序效率和资源利用率。右值引用可以更有效地处理临时对象,最大限度地减少不必要的开销。这些改进可以缩短执行时间并提高资源管理效率,从而提高 C++ 程序的整体性能。

C++11 中的语法简化和表达能力增强使代码更具可读性和可写性。Lam bda 表达式支持内联函数定义,提高了代码清晰度并减少了对辅助函数的需求。可变参数模板可以灵活地处理可变数量的模板参数,从而实现更通用和可重用的代码。基于范围的 for 循环为迭代容器提供了更直观的语法。这些改进使开发人员能够编写更具表现力、更高效和更灵活的代码,最终提高质量。

本章和接下来的两章讨论了这些改进中的大部分。未涵盖重大改进和增强,但可以在许多书籍、课程和互联网搜索中找到。C++11 标准中的增强范围非常值得研究,因为它们为 C++17 和 C++20 标准提供了基础,其中添加了增强、添加、弃用、删除和新语言功能。C++ 远未消亡,随着时间的推移,它会不断改进。

现代 C++ 无法防止开发人员犯错,但它支持一些使错误变得更加困难的功能。例如,基于范围的 for 循环不受 off-by-one 错误的影响。以下错误中指出的优点是旨在改进语言以解决经典错误的新功能。虽然已经添加了几个功能,但即使是很小的一个子集也会大大有利于开发过程。以下问题列表中讨论了其中一些功能。这里将讨论的还有很多,但本章中介绍的功能解决了前现代 C++ 中的一些限制的最基本功能。








