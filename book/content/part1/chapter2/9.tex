这个错误主要涉及正确性和有效性。共享指针比原始指针成本更高,当了解了该方法,代码的可读性会大大增强。

\mySamllsection{问题}

有几种数据结构,需要处理以各种方式排列的动态分配实体。这些动态资源最好使用 RAII 进行管理,当需要多次引用实体时,了解何时以及谁应该释放资源会变的很复杂。

清单 2.17 是开发人员的典例,他在大学课堂上学会了使用指针,并实现了双端队列。代码运行良好,但在管理实体资源时不鼓励使用原始指针。如果犯了错误,内存泄漏无法避免。此外,在抛出异常的应用程序中,管理的要求会更高。

\filename{清单2.17 使用原始指针实现的双端队列(deque)}

\begin{cpp}
class Node {
public:
  Node(int value) : data(value), prev(NULL), next(NULL) {}
  int data;
  Node* prev;
  Node* next;
};

class Deque {
private:
  Node* front;
  Node* back;
public:
  Deque() : front(NULL), back(NULL) {}
  ~Deque() { // 1
    while (front != NULL) {
      Node* temp = front;
      front = front->next;
      delete temp;
    }
  }
  void push_front(int value) {
    Node* newNode = new Node(value);
    if (front == NULL)
      front = back = newNode;
    else {
      newNode->next = front;
      front->prev = newNode;
      front = newNode;
    }
  }
  void push_back(int value) {
    Node* newNode = new Node(value);
    if (back == NULL)
      front = back = newNode;
    else {
      newNode->prev = back;
      back->next = newNode;
      back = newNode;
    }
  }
  void pop_front() {
    if (front != NULL) {
      Node* temp = front; // 2
      front = front->next;
      if (front != NULL)
        front->prev = NULL;
      else
        back = NULL;
      delete temp; // 3
    }
  }
  void pop_back() {
    if (back != NULL) {
      Node* temp = back; // 2
      back = back->prev;
      if (back != NULL)
        back->next = NULL;
      else
        front = NULL;
      delete temp; // 3
    }
  }
};

int main() {
  Deque deque;
  deque.push_front(3);
  deque.pop_front();
  return 0;
}
\end{cpp}

{\footnotesize
注释1:用于消除剩余动态实体的强制析构函数

注释2:对单个节点的两个引用

注释3:删除过期实体
}

\mySamllsection{分析}

要将节点放入双端队列或从双端队列中移除节点,至少需要两个指针来引用该节点。如果使用其他指针(索引队列),可多个指针引用单个节点。原始指针为多个指针引用单个节点提供了一种简单的方法,但这种方式使节点的管理变得复杂。

需要删除多个节点时,每个需要删除的节点都必须正确编码,否则很有可能出现内存泄漏,所以析构函数和这些单独的删除是一种认知负担。首先,必须正确设计和编码双端队列。其次,强制双端队列支持资源管理。这种方法会让代码变得更复杂、更难懂。

\mySamllsection{解决}

消除管理动态资源可以解决其中一个问题。现代 C++ 使用 std::shared\_ptr 来管理资源,尽可能使用 std::make\_shared,而非 new 操作符。共享指针使用 RAII 模式,可对单个资源进行多次引用。std::shared\_ptr 通过跟踪对资源的引用计数来管理资源。每次复制或分配时计数都会增加,每次超出范围的共享指针时计数都会减少。当计数达到0时,资源将删除。因此,不需要用户编写管理代码。太棒了!

std::unique\_ptr 成本很低,应该用于不需要共享的资源。std::shared\_ptr 成本相对较高,应该只在需要共享时使用。首先,共享指针(参见清单 2.18)比独占指针大,因其需要维护一个计数器。因此,只在必要时使用相应的智能指针,并尽可能优先使用独占指针。

\filename{清单2.18 使用 shared\_ptr 实现的双端队列(deque)}

\begin{cpp}
class Node {
public:
  Node(int value) : data(value) {}
  int data;
  std::shared_ptr<Node> prev; // 1
  std::shared_ptr<Node> next; // 1
};

class Deque {
private:
  std::shared_ptr<Node> front;
  std::shared_ptr<Node> back;
public:
  Deque() : front(nullptr), back(nullptr) {} // 2
  void push_front(int value) {
    std::shared_ptr<Node> newNode = std::make_shared<Node>(value);
    if (front == nullptr)
      front = back = newNode;
    else {
      newNode->next = front;
      front->prev = newNode;
      front = newNode;
    }
  }
  void push_back(int value) {
    std::shared_ptr<Node> newNode = std::make_shared<Node>(value);
    if (back == nullptr)
      front = back = newNode;
    else {
      newNode->prev = back;
      back->next = newNode;
      back = newNode;
    }
  }
  void pop_front() {
    if (front != nullptr) {
      front = front->next;
    if (front != nullptr)
      front->prev = nullptr;
    else
      back = nullptr;
    }
  }
  void pop_back() {
    if (back != nullptr) {
      back = back->prev;
      if (back != nullptr)
        back->next = nullptr;
      else
        front = nullptr;
    } // 2
  }
};

int main() {
  Deque deque;
  deque.push_front(3);
  deque.pop_front();
  return 0;
}
\end{cpp}

{\footnotesize
注释1:shared\_ptr 会隐式初始化

注释2:不需要析构函数和删除函数
}

std::shared\_ptr 的另一个优点是不需要显式初始化,在构造时即初始化为 nullptr。

使用共享指针时会出现一个问题——循环引用,即一个实体使用共享指针引用另一个实体,反之亦然。共享指针会跟踪对其共享实体的引用数量。每次构造引用它的共享指针时,计数都会增加,每次销毁共享指针时,计数都会减少。如果计数达到零,则删除共享资源;但在循环引用中,计数可能永远不会达到零,从而导致资源泄漏。

清单 2.18 中的代码,很容易复现循环引用问题。如果遇到这种情况,请考虑使用 std::weak\_pointer 来打破循环。弱指针 的工作方式类似于共享指针,但不共享资源的所有权。这种类型的指针必须在取消引用之前,转换为 std::shared\_ptr 。如果转换成功,则创建一个有效的指针;如果转换失败,则转换返回初始化为 nullptr 的共享指针。

虽然代码演示了双端队列,但标准模板库提供了自己的版本,称为 std::deque。请使用此双端队列,尽可能避免编写自己的双端队列,除非出于学习练手的目的!

\mySamllsection{建议}

\begin{itemize}
\item
使用智能指针无需管理动态资源,可将注意力集中在真正的问题上。

\item
当动态资源需要多个引用时,请使用 std::shared\_ptr;可使用 std::weak\_ptr 打破循环引用。

\item
请确保在取消引用之前转换 std::weak\_ptr。

\item
大多数情况下仍倾向于使用 std::unique\_ptr。
\end{itemize}

































