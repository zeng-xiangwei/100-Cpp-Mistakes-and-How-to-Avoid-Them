此错误主要针对可读性,但可能会提高效率。许多编程语言能在类定义中初始化实例变量;C++ 直到 C++11 才可以。

\mySamllsection{问题}

大多数开发人员发现,编写带有默认值的构造函数很方便。这些值通常很有意义,提供它们的唯一原因是默认值对于特定实例并不正确。

通常,开发人员会在构造函数的参数列表中使用默认值,如清单 2.9 所示。这种方法效果很好,但存在可读性问题。假设类中有多个构造函数,并且有许多实例变量。其中一些构造函数会默认很少更改的值,但读者可能需要帮助来确定给定实例变量的值。清单 2.9 中的代码易于阅读,请考虑它在具有4、5个或更多参数的类。作为示例,来阅读此代码,看看其中几个参数的默认值:

\begin{cpp}
Demo(int n=0, double d=1.0, double e=0.0, double f=10.5, bool b=false) :
n(n), d(d), e(e), f(f), b(b) {}
\end{cpp}

默认值应限制在可读且易懂的列表中;参数过多则表明类的责任过重,应将其重构为多种数据类型。以下代码没有犯此错误,但仍需改进可读性。

\filename{清单2.9 在参数列表中使用默认值}

\begin{cpp}
class Complex {
private:
  double real;
  double imag;
public:
  Complex(double r=0, double i=0) : real(r), imag(i) {} // 1
  double getReal() const { return real; }
  double getImag() const { return imag; }
};

int main() {
  Complex c1;
  Complex c2(3);
  Complex c3(-2, -2);
  return 0;
}
\end{cpp}

{\footnotesize
注释1:使用默认参数
}

\mySamllsection{分析}

虽然这段代码一切正常,但当提供多个构造函数时,可读性会有所下降。此外,编写的代码越多,引入错误的可能性就越大。如果两个或多个构造函数中的默认值参数不匹配,读者将很难确定实例变量的正确值。

\mySamllsection{解决}

现代 C++ 提供了一种不依赖默认参数值的替代方案。考虑到可读性,其他一些语言提供默认值,C++ 添加了类内初始化器。类中声明的每个实例变量都可以分配一个默认值,方法是使用赋值操作符或括号初始化形式(称为 统一初始化)。此形式可避免歧义和自动窄化转换,并且更加一致。

\filename{清单2.10 使用类内初始化以提高可读性}

\begin{cpp}
class Complex {
private:
  double real = 0; // 1
  double imag{0}; // 2
public:
  Complex() {}
  Complex(double r, double i) : real(r), imag(i) {}
  double getImag() const { return imag; }
};

int main() {
  Complex c1; // 3
  // Complex c2(3); // 4
  Complex c3(-2, -2);
  return 0;
}
\end{cpp}

{\footnotesize
注释1:使用赋值操作符进行类内初始化

注释2:使用带括号的类内初始化

注释3:使用默认值

注释4:错误,不知道该值打算初始化哪个成员变量
}

如果构造函数为类内初始化变量提供了一个值,则将使用所提供的值进行初始化。如果没有提供值,则使用类内的值。

\mySamllsection{建议}

\begin{itemize}
\item
选择类内初始化来本地化实例变量的默认值。

\item
使用传统 C++ 时,才在构造函数参数列表中使用默认值。
\end{itemize}
