这种错误会影响效率,并且会影响可读性。在派生类中实现虚函数,提供自定义继承行为的方法。错误的派生虚函数已经是一个很大的问题,所以标准化委员会同意为编译器添加功能,来避免这种情况发生。

\mySamllsection{问题}

包含虚函数的类旨在被继承,virtual 关键字表明了这一意图。在派生类中实现的虚函数,允许在基类中修改行为(或完全替换),这种方法是多态行为的核心。

在大型代码库中,基类和派生类可能相距甚远。如果距离很远,要知道某个方法是否是虚拟的就很困难了。派生类不必在其函数声明中重复virtual 关键字,但这种重复是确定给定方法为虚的合理方式。但这种文档无法解释虚函数的最基类版本在哪里,开发人员只能探索类型层次结构才能弄清楚。

清单 2.5 展示了一个快速而愉快的尝试,试图在周五晚上添加一些功能。代码编译成功,开发人员很满意,代码发布成功。遗憾的是,下周,客户抱怨 Square 对象无法正常工作。开发人员很惊讶,因为一切“看起来都很好”。

\filename{清单2.5 看起来不错且编译无误的代码}

\begin{cpp}
struct Shape {
  std::string type;
  explicit Shape(const std::string& t) : type(t) {}
  virtual double area() const { return 0; }
};

struct Square : public Shape {
  double side;
  Square(double s) : Shape("square"), side(s) {}
  double aria() const { return side*side; } // 1
};

const static double pi = 3.1415927;
struct Circle : public Shape {
  double radius;
  Circle(double r) : Shape("circle"), radius(r) {}
  double area() const { return pi * radius * radius; }
};

int main() {
  Square s(2);
  Circle c(2);
  std::vector<Shape*> shapes;
  shapes.push_back(&s);
  shapes.push_back(&c);
  for (int i = 0; i < shapes.size(); ++i)
    std::cout << shapes[i]->area() << '\n';
  return 0;
}
\end{cpp}

{\footnotesize
注释1:优美文字中也可能会有错别字。
}

\mySamllsection{分析}

问题很简单:一个手误。因为没有犯任何语法错误,编译器会将新函数添加到类中。但开发人员应该注意,成功编译并不意味着正确重新定义了相应虚函数。

错误的派生类虚函数在语法上没有错误——至少不是因为命名错误!而开发人员是否打算重新定义虚函数,编译器并不在乎。

\mySamllsection{解决}

C++11 标准添加了一个新关键字,称为 override。虚函数的问题非常普遍,必须采取一些措施来应对。这个新关键字添加到方法的参数列表之后,该方法应该是虚函数的重新定义。当编译器发现带有 override 关键字的方法,与某些基类中的虚函数不匹配,则会报错。这个关键字不是必需的,但其实现了两个重要的功能。首先,记录了此类重新定义了虚函数,从而提高了可读性。其次,编译器验证该方法是否重新定义了已知的虚函数。

因为在 aria 函数后添加了关键字,所以清单 2.5 中的代码无法通过编译。当编译器卡住时,开发人员可以看到错误;很快,问题就解决了。周五晚上的啤酒味道从来没有这么好过。

\begin{myTip}{}
override 从技术上讲是一个上下文关键字,而不是常规关键字。仅在特定上下文中才是 关键字(例如,在方法的参数列表之后) 。在其他上下文中,它是有效的(但不推荐!)标识符。之所以做出这种区分,是因为 override 在许多现有代码库中都作为标识符存在。标准化委员会对此非常友善。
\end{myTip}

\filename{清单2.6 使用 override}

\begin{cpp}
struct Shape {
  std::string type;
  Shape(const std::string& t) : type(t) {}
  virtual double area() const { return 0; }
};

struct Square : public Shape {
  double side;
  Square(double s) : Shape("square"), side(s) {}
  double area() const override { return side*side; } // 1
};

struct Circle : public Shape {
  double radius;
  Circle(double r) : Shape("circle"), radius(r) {}
  double area() const override { return
    std::numbers::pi * radius * radius; } // 1
};

int main() {
  Square s(2);
  Circle c(2);
  std::vector<Shape*> shapes;
  shapes.push_back(&s);
  shapes.push_back(&c);
  for (auto& shape : shapes)
    std::cout << shape->area() << '\n';
  return 0;
}
\end{cpp}

{\footnotesize
注释1:添加override,让编译器检查虚函数的情况
}

\mySamllsection{建议}

\begin{itemize}
\item
使用 override 关键字来表明重新定义继承虚函数的意图。

\item
使用 override 让编译器强制重新定义虚函数。

\item
override 是一个上下文关键字,但请避免在其他地方使用它——会使代码变得复杂。
\end{itemize}





















