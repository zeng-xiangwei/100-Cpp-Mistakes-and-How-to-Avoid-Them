这个错误会影响效率。可读性通常也会得到增强,但当开发人员希望知道所使用的确切数据类型时,这些信息是不透明的。

\mySamllsection{问题}

旧习难改。以下清单中的代码展示了索引循环和测试搜索结果的选择语句的典型方法。这些机制的传统用法已将其形式刻在我们的大脑中,使它们成为我们的第二天性。然而,传统并不总是意味着正确性。

\filename{清单3.3 典型的基于索引的循环和测试变量}

\begin{cpp}
int main() {
  std::vector<int> values;
  for (int i = 0; i < 10; ++i) // 1
    values.push_back(i*2);
  int key = 4;
  std::vector<int>::iterator it =
  std::find(values.begin(), values.end()) // 2
  if (it != values.end())
    std::cout << key << " was found\n";
  else
    std::cout << key << " was not found\n";
  return 0;
}
\end{cpp}

{\footnotesize
注释1:带有循环控制变量的典型基于索引的循环

注释2:数据类型正确,但是直观吗?
}

\mySamllsection{分析}

许多解决日常问题的传统方法(例如循环和变量类型)反映了语言的原始状态。编译器过去不太擅长与开发人员交互,并且依赖于静态方法来声明变量 - 编译器要么批准要么拒绝开发人员的选择。基于索引的循环非常典型,甚至不会引起注意,但其循环控制变量在技术上是错误的。

容器的索引始终从零开始,一直增加到容器的长度(但不包括长度) 。在这种情况下,负值是不正确的,但 int 变量可以模拟正值、零值和负值。如果我们过于吹毛求疵,整数不可能是正确的数据类型。用于测试 find 算法结果的变量要求我们了解容器、其元素及其迭代器 类型——这三部分知识需要跟踪。更糟糕的是,这些事实与正在解决的问题无关;它们是开销。开销会对效率产生负面影响并影响表达能力。

\mySamllsection{解决}

现代 C++ 追随其他更现代的语言的脚步,为编译器提供了更多的智能和灵活性。由于编译器最终知道数据类型是否正确,它可以利用这些知识来确定理想的类型并将其插入到代码中。清单 3.4 解决了前面提到的两个问题。

首先,编译器确定循环控制变量的有效类型 — 理想情况下,它将是无符号的。不过,由于其初始化器为零,因此它将被推断为 int。无符号不仅会给出两倍的范围,而且还会排除任何负索引值。从数学上讲,它是正确的类型;实际类型是 size\_t,通常实现为unsigned long 或 unsigned long long。它保证可以表示任何对象的大小,因此请放心使用它。使用 auto 并让编译器确定其具体类型会更好。

其次,类型的复杂性可能会分散人们对所要解决的问题的注意力,并引入其他细微的错误。编译器确定正确数据类型的能力可以减轻这种复杂性并防止引入异常。使用 auto 关键字可以利用编译器推断正确数据类型的能力,更好的是,它可以将这些信息添加到代码中。开发人员专注于所要解决的问题,语言负责处理日常事务,这是一种协同作用——每个人都做自己最擅长的事情。

\filename{清单3.4 使用auto让编译器推断出正确的类型}

\begin{cpp}
int main() {
  std::vector<int> values;
  for (auto i = 0; i < 10; ++i) // 1
    values.push_back(i*2);
  int key = 4;
  if (auto it = std::find(values.begin(),
          values.end(), key); it != values.end()) // 2
    std::cout << key << " was found\n";
  else
    std::cout << key << " was not found\n";
  return 0;
}
\end{cpp}

{\footnotesize
注释1:编译器总是为循环控制变量选择正确的数据类型

注释2:让编译器找出确切的类型,这样我们就不用记住那么多了
}

auto 关键字有一个注意事项:倒计时循环有些棘手。以下代码展示了一种以相反顺序显示元素的简单方法。

\filename{清单3.5 对auto的一个微妙的误用}

\begin{cpp}
int main() {
  std::vector<int> values;
  for (int i = 0; i < 10; ++i)
    values.push_back(i*2);
  for (auto i = values.size() - 1; i >= 0; --i) // 1
    std::cout << values[i] << '\n';
  return 0;
}
\end{cpp}

{\footnotesize
注释1:反向实现循环
}

当此代码在我的系统上运行的时候,它会崩溃,因为出现段错误。为什么?请记住,循环控制变量的推导类型 size\_t 是 unsigned。是否有不等于或大于零的无符号值?没有。因此,当 i 变为零时,它的下一个值是数据类型的最大值 — 当然远远超出了容器的边界。这种情况最容易解决的方法是将数据类型改为 int 而不是 auto。在这种情况下,size\_t 是错误的;这是类型推导可能引入的那些细微错误之一。由于应用程序二进制接口 (ABI) 的稳定性和向后兼容性,没有标准会改变这一点。算了。

另一个影响性能的细微问题出现在使用 auto 推断值的类型时,例如以下情况:

\begin{cpp}
auto name = student->getName();
\end{cpp}

name 变量是学生姓名的副本,但引用会更好。引用是代表学生姓名的 std::string 的别名,而不是其副本,从而节省了对复制构造函数的调用。代码如下所示:

\begin{cpp}
auto& name = student->getName();
\end{cpp}

这种区别也发生在使用 auto 在基于范围的 for 循环中推导循环控制变量时。变量始终是元素的副本。虽然这鼓励元素的只读特性,但如果元素类型大于指针,则效率会很低。此代码演示了对 students 列 表进行迭代的首选方式:

\begin{cpp}
for (auto& student : students) ...
\end{cpp}

\mySamllsection{建议}

\begin{itemize}
\item
使用 auto 进行许多变量声明。

\item
使用 auto 关键字可使模板受益匪浅。

\item
在数据类型大于指针的情况下,请自由使用 auto 引用,以防止复制。

\item
开发人员很少需要知道确切的类型 - 如果需要,应将其拼写出来。
\end{itemize}
