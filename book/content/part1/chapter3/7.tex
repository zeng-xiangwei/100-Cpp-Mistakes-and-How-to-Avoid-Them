这个错误关注的是效率。创建结构体或类来保存多个值,会分散开发人员对解决问题的注意力。

\mySamllsection{问题}

一个典型的问题是将多个数据类型聚合到一个单元中,以便于操作。通常使用 struct 或 class,这需要开发人员创建代码来定义聚合。这种方法已经行之有效多年,但对于普通旧数据 (POD) 来说,当结构体仅在少数地方使用时,这往往有些矫枉过正。

清单 3.14 中需要对学生的三个属性进行建模—— 姓名、年龄和平均成绩。代码中还必须定义几个类似的类来承载其他聚合组,增加结构数量并使用不同的数据类型。这种方法的一个明显优势是,可以通过字段名称访问 POD——稍后会详细介绍。

\filename{清单3.14 将简单POD定义为结构体}

\begin{cpp}
struct Student { // 1
  std::string name;
  int age;
  double gpa;
  Student(const std::string& n, int a, double g) : name(n), age(a), gpa(g) {}
};

// other well-defined PODs
struct Employee {};
struct Teacher {};
struct Admin{};

int main() {
  Student s("Susan", 23, 3.85); // 2
  std::cout << "student " << s.name << ", " << s.age << " years old, carries a "
            << s.gpa << '\n'; // 3
  return 0;
}
\end{cpp}

{\footnotesize
注释1:定义简单的 POD

注释2:初始化POD

注释3:使用POD
}

\mySamllsection{分析}

每个聚合都需要定义一种新的数据类型,必须定义实例变量及其数据类型。最简单的方法是使用构造函数来初始化实例。通过实例变量名访问变量,这是一种自然而有效的方式。但是,类似代码的重复,以及编写每个新 POD 的负担,使这种方法繁琐且容易出错。此外,对数据类型的理解会因代码量、定义之间的距离,以及类似部分的重复而减弱。

\mySamllsection{解决}

如清单 3.15 所示,添加了 using 关键字以简化数据类型名称的创建及其定义。

此代码中的关键是元组,tuple。std::tuple 模板通过枚举其类型来声明多个不同类型的实例变量,它们没有名称。结构体中的这种差异最大限度地减少了命名和编码。实例的初始化与结构版本完全一样,但无需提供构造函数 — 另一种节省时间和减少混乱的手段。

最后一个区别是负面的;访问字段不是通过名称,而是通过索引。每个实例变量都按规范顺序进行索引,从索引零开始。给定一个索引,std::get 函数模板有助于确定要访问哪个字段。

\filename{清单3.15 将简单POD定义为元组}

\begin{cpp}
using Student = std::tuple<std::string, int, double>; // 1

// other well-defined PODs
// using Employee = std::tuple<...>;
// using Teacher = std::tuple<...>;
// using Admin = std::tuple<...>;

int main() {
  Student s("Susan", 23, 3.85); // 2
  std::cout << "student " << std::get<0>(s) << ", " << std::get<1>(s)
      << " years old, carries a " << std::get<2>(s) << '\n'; // 3
  return 0;
}
\end{cpp}

{\footnotesize
注释1:低开销的POD定义方法

注释2:初始化POD

注释3:使用POD
}

不需要类的完整语义时,std::tuple 模板提供了一种紧凑且可用的方法,来定义数据聚合;这种方法特别适用于 POD。模板可用于进一步概括元组。例如,可以使用此替代方法来定义和使用 Student:

\begin{cpp}
template <typename T1, typename T2, typename T3>
using Student = std::tuple<T1, T2, T3>;
\end{cpp}

初始化将是

\begin{cpp}
Student<std::string, int, double> s("Susan", 23, 3.85);
\end{cpp}

对字段的访问保持不变。

这里有一个警告:如果 Student 和 Employee 定义相同,则它们是同一类型。任何使用这种 Student 的函数或容器也可以使用Employee — 这不是里氏替换原则,而是一个严重的设计缺陷。请仔细考虑如何使用。

关于以不同方式使用元组的最后一点意见是,当函数需要返回多个值时,请考虑使用 std::tuple 模板。此应用程序将是考虑使用 using 关键字的模板的一个很好的理由。输出参数应替换为元组以增强可读性。

\mySamllsection{建议}

\begin{itemize}
\item
对于简单结构(通常是 POD),使用 std::tuple 来简化编码并方便阅读。

\item
考虑模板化元组,以实现更通用的用途。

\item
使用元组返回多个值。
\end{itemize}









