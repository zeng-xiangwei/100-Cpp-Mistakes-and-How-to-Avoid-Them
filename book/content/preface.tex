学习C++的核心目标在于运用其独特的语言特性,以解决特定环境下的问题。无论是大学教育,还是工作场所对初级开发者的指导,虽然场景不同,但所使用的语言基础是相同的。可以将C++视为开发者在最底层所使用的通用语言。然而,在实际应用中,设计模式、常规使用方法、具体问题领域的细节,以及公司内部流程则为更高层次的沟通方式,而这些高层次的内容才是最为关键的部分。正如Alan Turing所证明的,“任何一台计算机都能够计算其他计算机能够解决的所有可解问题,只不过实现的方法和所需的时间可能有所不同”。同样地,任何编程语言都可以处理C++所能解决的所有问题。

这并非旨在批评C++(或者其他的编程语言),而是强调这样一个事实:即使是在企业环境中使用C++,对于公司整体发展方向和解决的问题而言,几乎不会产生影响。成为一名资深开发者的技能,远比掌握某一特定编程语言的知识重要得多。

尽管如此,为什么这本书仍然专注于C++呢?在此之前,我们已经确定各位读者在使用C++的工作环境中生存。基于对这门语言的兴趣与技能,将有机会运用自己的知识和技能来应对公司关注的问题。换句话说,你将通过C++来完成某项任务。

因此,了解如何识别C++中常见的错误,可以帮助你减少这些错误的发生。更重要的是,避免再次犯同样的错误。如果必须为一个特定方法主导的代码库工作,那么就失去了以最具表现力和最合适的方式使用C++的机会。例如,许多字符串处理方案仍然采用C语言风格的习惯用法,如strcat、strstr和strcpy。如果这种做法在团队中普遍存在,你也可能会倾向于甚至被迫继续沿用这些方法。其实,应该避免使用这些函数,而优先考虑C++提供的更安全的方法。

C++是一种极其灵活的语言,允许开发者执行机器所能完成的任何操作。许多较新的编程语言通过抽象隐藏了大部分机器细节和复杂性,包括Go和Rust。相比之下,C++提供了操纵机器特性的最低粒度,使其成为处理底层操作的理想选择。

然而,这种灵活性和精细控制并非没有代价,无法像使用Python或Java那样以“宽松”的方式编写C++代码。例如,C++并没有提供默认的垃圾回收机制来自动管理内存和其他资源,开发者必须手动处理。在这种资源管理过程中出现的错误,往往会对程序的性能和正确性造成严重影响。引用《蜘蛛侠》(2002)电影中的台词似乎非常恰当,其中Peter Park的叔叔Ben警告过Peter Park:“能力越大,责任越大。”这句话完全可以表达C++的开发哲学。

我对C++的兴趣起源于Bjarne Stroustrup博士开发的从C++到C的转译器CFront。在此之前,我已经学习并使用过C语言,并一直在寻找机会实践C++项目。最终,在20世纪80年代末,随着工作站革命的到来,我开始正式学习C++。随后大约在1995年,Java的出现中断了我的C++之旅。直到2020年代初,我又重新开始用C++进行项目开发。同时,我也在大学教授编程课程,并开设了许多关于C++的课程。

这两个阶段的教学经历中,我逐渐意识到C++的一些缺点、经典问题及惯用法的存在,同时注意到现代C++的特性应用相对较少;很多教科书几乎没有提及C++的现代特性。因此,我决定以书籍的形式分享我的经验和发现。我希望,学生和从业者能够受到鼓舞,去审视现有的代码,理解其中存在的问题,并探索改进的方法。如果他们能够成功避开这些陷阱,则在学习和实践中就已经领先于许多人。
