这个错误主要针对性能和效率,初始化列表可增强初始化实例变量的性能。

\mySamllsection{问题}

清单 6.11 中的代码,显示了尝试从提供的参数构建实例。完整“标题”是title、first、middle 和 last 名称的组合。在构造函数中用这些名称初始化非常合理,所有其他变量都从提供的值初始化。这种情况下,可以在访问器中构建标题。尽管如此,为了挽救轻微的性能差异(假设从未访问过标题),所做的努力是微不足道,并迫使开发人员编写更多代码。这种方法影响了可读性,却没有提供足够的好处。

当代码在我的系统上运行的时候,会抛出一个 std::bad\_alloc 异常,开发人员会修复这个错误。这个结果是这个问题的一个可能的影响,系统可能会有不同的行为。无论如何,代码存在导致未定义行为的问题。如果碰巧这个代码在给定的系统上运行,则会产生奇怪的输出,可能很难调试。

\filename{清单6.11 混淆初始化顺序}

\begin{cpp}
struct Person {
  std::string full;
  std::string first;
  std::string middle;
  std::string last;
  std::string title;
  Person(std::string f, std::string m, std::string l, std::string t) :
    first(f), middle(m), last(l), title(t),
    full(title + ' ' + first + ' ' + ' ' + middle + ' ' + last ) {} // 1
};

int main() {
  Person judge("Hank", "M.", "Hye", "Hon.");
  std::cout << judge.full << '\n';
  return 0;
}
\end{cpp}

{\footnotesize
注释1:构建完整的标题
}

\mySamllsection{分析}

这段代码大部分都没问题,但请注意实例变量的声明顺序和初始化顺序。首先声明了 full 实例变量,然后是其余变量。因为各个实例变量(似乎)已经初始化了,所以开发者对此进行了推理,并决定最后初始化这个实例变量。

开发人员在这里犯了一个错误;尽管完整的实例变量似乎最后才初始化,但编译器会编写代码按声明顺序初始化变量。因此,full 会先初始化,然后才是其他变量。这些未初始化变量的连接,会导致异常或其他不良行为。

理解了这个强制顺序之后,开发人员可以通过将代码放在构造函数主体中来控制初始化。清单 6.12 显示了结果。

\filename{清单6.12 将初始化推入构造函数体}

\begin{cpp}
struct Person {
  std::string full;
  std::string first;
  std::string middle;
  std::string last;
  std::string title;
  Person(std::string f, std::string m, std::string l, std::string t) { // 1
    first = f; middle = m; last = l; title = t;
    full = title + ' ' + first + ' ' + ' ' + middle + ' ' + last;
  }
};

int main() {
  Person judge("Hank", "M.", "Hye", "Hon.");
  std::cout << judge.full << '\n';
  return 0;
}
\end{cpp}

{\footnotesize
注释1:将所有初始化都放在构造函数主体中(性能较差)
}

此解决方案有效,但需要提高性能。这可能并不明显,但所有变量仍然使用初始化列表形式进行初始化

\begin{cpp}
Person(std::string f, std::string m, std::string l, std::string t) :
  full(), first(), middle(), last(), title() {}
\end{cpp}

编译器确保所有类实例都使用其默认构造函数进行初始化,所有实例变量都初始化为初始化列表中的默认值。初始化之后,执行构造函数主体,并将参数值分配给现有实例。这是每次操作成本的两倍(暂时不要挑剔复制构造函数和复制赋值操作符之间的性能差异) 。不过,这种方法并非最优。

\mySamllsection{解决}

由于编译器将使用初始化列表形式,来初始化此类声明中的每个基于类的实例变量(而非基元变量)。如果顺序很重要,理想的方法是按照变量应初始化的顺序对其进行排序。此示例对顺序很敏感,但应尽可能避免这种情况。如果顺序不敏感,请选择有意义的顺序(按字母顺序、按类型分组等)。以下代码考虑了顺序,并确保只有在其所有组成部分都完全初始化后,才初始化 full。

\filename{清单6.13 严格使用初始化顺序}

\begin{cpp}
struct Person {
  std::string first;
  std::string middle;
  std::string last;
  std::string title;
  std::string full;
  Person(std::string f, std::string m, std::string l,
         std::string t) : // 1
    first(f), middle(m), last(l), title(t),

  full(title + ' ' + first + ' ' + ' ' + middle + ' ' + last ) {}
};

int main() {
  Person judge("Hank", "M.", "Hye", "Hon.");
  std::cout << judge.full << '\n';
  return 0;
}
\end{cpp}

{\footnotesize
注释1:重新排序变量,以确保首先初始化组件部分
}

\mySamllsection{建议}

\begin{itemize}
\item
如果可能的话,避免使用复合实例变量;如果不能,请对它们进行排序,以便首先初始化所有组成部分。

\item
对所有实例变量使用初始化列表;如果不使用它们,编译器会生成它们。

\item
确保编译器警告已打开;这是最早、最广泛检测到的问题方法(在编译时始终使用 -Wall)。
\end{itemize}













