这个问题影响正确性。在大多数情况下,容器是数组的首选替代方案。然而,这一事实并不能消除使用容器的问题。

\mySamllsection{问题}

通常,编程问题涉及通常处理或处理的数据组。数组是将单个数据元素作为一个单元保存和管理的经典解决方案;但是,它们应该很少使用。标准模板库提供一些容器比数组更加智能,并且提供了值得使用的功能。

假设我们的开发人员想要打印某个活动的人员列表。其中会混合使用Persons 和 Employees,但必须全部列出。清单 6.14 中的代码展示了 这一尝试。由于其所有权语义,开发人员被鼓励在编程期间使用自动指针 (auto\_ptr)。大胆前进,代码被开发和测试。其结果可能会更令人鼓舞。

\filename{清单6.14 滥用多态元素的容器}

\begin{cpp}
struct Person {
  std::string name;
  Person(const std::string& n) : name(n) {}
  virtual std::string toString() { return name; }
};

struct Employee : public Person {
  double salary;
  Employee(const std::string& n, double s) : Person(n), salary(s) {}
  std::string toString() {
    std::stringstream ss;
    ss << Person::name << " gets paid " << salary;
    return ss.str();
  }
};

int main() {
  Person p("Sue");
  Employee e("Jane", 123.45);

  std::vector<Person> people;
  people.push_back(p);
  people.push_back(e);
  for (int i = 0; i < people.size(); ++i)
    std::cout << people[i].toString() << '\n'; // 1

  std::vector<std::auto_ptr<Person> > persons;
  // persons.push_back(p); persons.push_back(e); // 2

  std::vector<Person> peeps; // 3

  peeps.push_back(p);
  for (int i = 0; i < peeps.size(); ++i)
    std::cout << peeps[i].toString() << '\n';

  std::vector<Employee> emps; // 3
  emps.push_back(e);
  for (int i = 0; i < emps.size(); ++i)
    std::cout << emps[i].toString() << '\n';
  return 0;
}
\end{cpp}

{\footnotesize
注释1:多态行为不起作用

注释2:这无法编译;它具有不同的元素类型

注释3:单独的vector解决了该问题
}

\mySamllsection{分析}

尝试使用 auto\_ptr 向量无法编译。开发人员对此代码进行了评论,希望以后对其进行研究。由于开发人员迫切希望代码能够正常工作,因此他们编写了单独的容器来保存每种数据类型。这种方法有效,但需要优化。

单个向量的问题在于它保存的是基类型的对象。每个元素只分配了足够容纳基类对象的空间,从而切断了派生类数据。使用数组解决同一问题时也会出现此问题。多态行为就是这样!

第二次尝试是使用新奇的 auto\_ptr 类型,它独占容器内的实例。这种方法很有意义,但会遇到麻烦。只有可复制构造和可分配的数据类型才能插入容器中。auto\_ptr 类型两者都不是。因为它支持独占所有权语义,所以无法将实例添加到向量中——新奇的想法就这么多。(现代 C++ 有一个更比使用 auto\_ptr 更好的方法;出于充分的理由,此类型在 C++11 中已被弃用)。开发人员的最后一次尝试解决了该问题;但是,如果没有单个容器的灵活性,解决方案可能会更好。

\mySamllsection{解决}

容器应包含值类型,可以是基元、不与派生类实例混合的类类型或指针(在本例中为原始指针)。现代 C++ 已经解决了原始指针问题,因此此建议仅应在前现代 C++ 中使用。选择原始指针作为容器的元素类型是为了保持开发人员想要的灵活性、消除切片问题并允许复制构造和赋值。以下代码展示了解决方案并展示了开发人员的所有首选行为。

\filename{清单6.15 为多态元素正确使用容器}

\begin{cpp}
struct Person {
  std::string name;
  Person(const std::string& n) : name(n) {}
  virtual std::string toString() const { return name; }
  virtual ~Person() {}
};

struct Employee : public Person {
  double salary;
  Employee(const std::string& n, double s) : Person(n), salary(s) {}
  std::string toString() const {
    std::stringstream ss;
    ss << Person::name << " gets paid " << salary;
    return ss.str();
  }
};
int main() {
  Person p("Sue");
  Employee e("Jane", 123.45);
  std::vector<Person*> people;
  people.push_back(&p); // 1
  people.push_back(&e); // 1
  for (int i = 0; i < people.size(); ++i)
    std::cout << people[i]->toString() << '\n'; // 2
  return 0;
}
\end{cpp}

{\footnotesize
注释1:每一个都可以添加;它们是相同的元素类型

注释2:多态行为有效
}

容器在使用多态元素时很有用,但必须正确管理。确保基类添加虚拟析构函数,因为容器在构造和析构期间会复制和删除元素。代码安全可确保运行时稳定性!

\mySamllsection{建议}

\begin{itemize}
\item
确保添加到容器中的任何内容都是可复制构造和可分配的;消除其中一个或两个的类不符合纳入条件。

\item
容器应保存值类型或指针类型,以防止切片或其他问题。

\item
如果可能,请考虑现代 C++ 改进。
\end{itemize}










