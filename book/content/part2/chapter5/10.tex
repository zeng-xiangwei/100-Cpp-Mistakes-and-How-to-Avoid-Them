这个错误与有效性和可读性有关;会影响正确性,如果是这样,会产生不利影响。C 提供了一个内置数据容器,数组是可以用一个名称寻址,并通过索引区分的元素序列。

\mySamllsection{问题}

许多编程问题都需要集合,无论是内置的还是用户编写的。如果对象是用户编写的并存储在数组中,则必须存在默认构造函数。在某些情况下,此构造函数有意义,但在许多情况下,它没有意义。毫无意义的默认构造函数表明需要更正或完成设计,但使用数组不是一种选择。

创建静态数组时,必须在编译时知道元素的数量。许多问题没有清楚地表达必须处理的元素数量,所以可以选择任意大小。如果开发人员的猜测太大,就会浪费空间;如果猜测太小,这通常会导致崩溃,甚至更糟。这里一个很好的选择是使用动态数组,代码可以在需要容器之前确定元素的数量。开发人员必须记住管理容器的内存——动态数组也需要默认构造函数。

无论基于数组的容器有多少元素,添加过多的元素都是可能的。使用动态数组允许开发人员通过分配较大的数组,并复制值来管理过短的数组。如果该数组被其他实体或对象引用,开发人员通常会感到意外,因为他们不知道分配的内存已转移。

最后,删除不再需要或有效的数组元素。使用数组的所有代码都必须清楚,哪些元素无效无效(或已删除)。这个问题会将知识分散到使用该数组的函数中,并使工作量重复。由于数组是哑对象,因此无法询问哪些元素有效。开发人员必须确定一种方案,通过某种方式将各个元素标记为无效——读者可能不清楚此代码的含义或为什么它会与某些元素发生冲突。以下代码演示了静态数组的简单情况,该数组强制 Person 类包含伪默认构造函数。

\filename{清单5.22 静态数组强制使用默认构造函数}

\begin{cpp}
struct Person {
  std::string name;
  int age;
  Person(const std::string& n, int a) : name(n), age(a) {}
  Person() : name("", 0) {}
};

int main() {
  Person people[3]; // 1
  Person suzy("Susan", 25);
  people[0] = suzy;
  Person anna("Annette", 32);
  people[2] = anna;
  std::cout << people[1].name << '\n';

  int count = 5; // assume this is computed
  Person* others = new Person[count]; // 1
  return 0;
}
\end{cpp}

{\footnotesize
注释1:每个元素可通过调用默认构造函数进行初始化
}

\mySamllsection{分析}

第一个数组是静态的;元素数量必须在编译时知道,此示例表明元素数量太少。尝试了另一种方法,即在确定元素数量后分配动态数组。这两种情况下,都需要默认构造函数,但实例变量没有合理的默认值。如果五个元素中有四个有意义地初始化,则缺少的元素仍将是“合法的” Person 元素;但将包含非法信息,因其数据不代表任何人。

\mySamllsection{解决}

用vector代替数组几乎总是正确的选择。 Stroustrup 博士(C++ 的发明者)推荐这种方法,所以我对此深信不疑。即使他没有这样说过,使用vector的理由本身也足够令人信服。

首先,vector是动态的,使用vector 在某些方面类似于数组。数组用于实现称为 支持数组 的vector。当支持数组填满,并且没有多余的元素时,就会发生神奇的事情。添加新元素时不会发生崩溃,vector将分配一个新的、更大的支持数组;从前一个数组复制元素;并将新元素添加到第一个未使用的索引。开发人员只需享受轻松使用的乐趣,并且不需要用户编写内存管理代码即可获得此结果。请不要假设vector不会滥用或使用而不会产生后果;错误的代码可能会导致异常!

其次,可以将元素推送到vector中,而无需考虑特定的索引值。如果索引有效,则可以添加元素。对于开发人员而言,删除元素就是删除该元素;无需设置哨兵来指示无效元素。

第三,如果对后备数组进行多次重新分配,则可能会出现严重的性能下降,这是由于元素被复制的次数所致。最好估算所需元素的数量,并调用 reserve 函数。reserve 函数会为该数量的元素分配足够的内存。选择正确的值意味着不会重新分配或浪费空间遇到这种情况时,即使猜测错误,vector也会正确运行。如果猜测值太低,则会发生重新分配;如果猜测值太高,则会浪费空间。但开发人员不必管理内存或为vector的机制担心。

以下代码改进了数组实现,使编码更加流畅,错误更少。开发人员不会凭空得到任何东西;只要明智地使用,vector就可以接近这个不可能的梦想。

\filename{清单5.23 使用vector替换笨拙的数组}

\begin{cpp}
struct Person {
  std::string name;
  int age;
  Person(const std::string& n, int a) : name(n), age(a) {}
};

int main() {
  std::vector<Person> people;
  Person suzy("Susan", 25);
  people.push_back(suzy);
  Person anna("Annette", 32);
  people.push_back(anna);
  std::cout << people[people.size()-1].name << '\n';

  int count = 5; // assume this is computed
  std::vector<Person> others;
  others.reserve(count);
  return 0;
}
\end{cpp}

\mySamllsection{建议}

\begin{itemize}
\item
大多数情况下,用vector代替数组。

\item
仔细阅读并理解使用vector 的空间和时间含义;许多情况下,不会出现问题,并且具有出色的性能。

\item
vector有许多可用的方法;研究它们以了解它们的能力和可能性。
\end{itemize}













