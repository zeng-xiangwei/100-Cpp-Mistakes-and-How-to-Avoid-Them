这个错误影响了效率和可读性。开发人员多年来一直在使用 C 语言的 FILE 对象,但它通常比预想的更复杂。

\mySamllsection{问题}

开发人员经常需要打开文件并读取文本行,而 C 语言的 FILE 对象已在 C++ 代码中广泛使用。以下清单中的代码显示了一种常见方法,即尝试打开文件;测试此假设是否准确;如果正确,则继续逐行读取文件。

\filename{清单5.7 使用FILE读取文本行}

\begin{cpp}
const int SIZE = 100; // 1
int main() {
  FILE* file;
  file = fopen("data.txt", "r");
  if (!file) { // 2
    std::cerr << "Error opening file\n";
    return 1;
  }
  char buffer[SIZE];
  while (!feof(file)) {
    if (!fgets(buffer, SIZE, file)) // 3
      break;
    std::cout << buffer;
  }
  return 0;
}
\end{cpp}

{\footnotesize
注释1:希望每行不超过 99 个字符

注释2:首次使用负逻辑

注释3:负逻辑,希望读取整行
}

\mySamllsection{分析}

清单 5.7 中的代码运行良好,实现了编写它的目的。代码在三个地方使用了负逻辑:测试是否成功打开文件、测试行尾条件以及测试是否已读取行。负逻辑总是比正逻辑更耗费认知。当负逻辑嵌套时,负载会增加,如本例所示,编写负逻辑会更加困难。看似自然的方法必须反转,并且结果流必须保存在短期记忆中。

代码的另一个问题是行长。代码假设每行的长度不会超过 99 个字符,这个假设合理吗?无论使用什么值,都存在这样的可能:要么太小,导致每行读取多次;要么太大,导致内存使用效率低下。

虽然从技术上来说这不是错误,但缓冲区定义应该大一个字符以预测行尾字符。假设一行文本恰好是 SIZE 个字符,则将在循环的下一次迭代中读取行尾字符。这种方法对性能有轻微影响,因为每行都应该进行一次读取操作,且SIZE 没有很好定义。

最后,FILE 对象保持打开状态。长期运行的程序如果存在此错误,可能会对系统资源产生负面影响,并可能导致资源耗尽。

\mySamllsection{解决}

这些问题可以使用 C++ 的 std::ifstream 对象来解决。清单 5.8 中没有负逻辑,更容易编写和阅读。缺乏负逻辑意味着代码读起来很简单。如果未打开输入文件,则失败测试读起来很自然。

此解决方案没有缓冲区;而是使用 std::string 对象。每行输入所需的大小都是 line 变量的长度。此外,在创建每行时,字符串中不存在先前的数据。执行部分读取时,缓冲区总是容易受到其他数据的影响。

std::ifstream 对象在 main 函数结束时超出范围。这一事实确保文件将关闭。这是 RAII 模式的一个示例,其中构造函数-析构函数对管理动态资源。可以关闭流,但没有必要。这种方法简化了写入和读取。开发人员必须确保记录自动销毁;否则,他们只能猜测会发生什么。标准模板库在这方面非常出色。

\filename{清单5.8 使用ifstream读取文本行}

\begin{cpp}
int main() {
  std::ifstream file("data.txt");
  if (file.fail()) { // 1
    std::cerr << "Error opening file\n";
    return 1;
  }
  std::string line;
  while (std::getline(file, line)) // 2
  std::cout << line << '\n';
  return 0;
}
\end{cpp}

{\footnotesize
注释1:积极逻辑使阅读和写作更容易

注释2:更积极的逻辑,不覆盖旧数据
}

std::ifstream 方法是一种读取文本数据,并测试流中是否存在影响其操作的错误的简单方法。此方法优于从 C 语言继承的 FILE* 方法。

\mySamllsection{建议}

\begin{itemize}
\item
使用 std::ifstream 和 std::ofstream 对象简化文件数据的读取和写入。

\item
尽可能使用正逻辑。
\end{itemize}


































