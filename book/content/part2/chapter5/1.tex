此错误分别影响正确性、可读性和有效性。C 要求在函数顶部声明所有变量;开发人员通常在那里初始化它们。

\mySamllsection{问题}

许多语言要求在函数顶部声明所有变量,有时也定义所有变量。这是对编译器的让步,而不是开发人员的需求,这种做法存在一些问题。首先,当所有变量都在函数顶部声明时,读者和开发人员必须参考顶部来查看名称、记住它们的含义,并查看变量的使用位置。这种方法需要一定的认知负荷,这可能会很繁重。其次,开发人员必须在使用变量之前确定变量的初始化;为方便起见,此初始化可能是一个简单的值(通常为零),但不一定反映后面的代码所依赖的正确值。但是,如果程序员花时间正确地初始化该值,读者可能要到很晚才明白为什么。维护程序员可能需要稍后更改此值,从而引入问题。最后,这些初始化值可能经常看起来像是神奇的数字;为什么 1 应该被初始化为 0、1 或(显然)任意值?

清单 5.1 演示了其中的一些问题。开发人员需要初始化 max 变量,但由于膝跳值为 0,由于初始化不当,它会计算出错误的答案。此外,pos 变量在循环之前的代码中使用,并保留了一些值。原始代码假设pos 将在(某处)初始化并在循环中正确使用。后来的开发人员忽略 了这个未记录的假设,并将“完美无缺”的变量用于其他目的。

\filename{清单5.1 在函数的顶部声明变量}

\begin{cpp}
int maximum(const std::vector<int>& values) {
  int max = 0; // 1
  int pos; // 2

  // assume code here that uses pos ...
  pos = 1; // 3
  // assume more code here...
  for (; pos < values.size(); ++pos) // 4
    if (values[pos] > max)
      max = values[pos];
  return max;
}

int main() {
  std::vector<int> values;
  values.push_back(1);
  values.push_back(-2);
  values.push_back(-3);
  std::cout << maximum(values) << '\n';
  return 0;
}
\end{cpp}

{\footnotesize
注释1:变量的初始化选择不当

注释2:没有初始化;假定这将在稍后发生

注释3:稍后添加的一些计算的结束值

注释4:使用 pos 时假定其值是有意义的
}

\mySamllsection{分析}

变量声明与变量使用之间的分离使得变量在预期使用之前会发生奇怪的事情。pos 变量对于一些额外的计算来说是一个不错的选择,但它处于不适合循环的状态。max 变量的初始化使用了一个通常正确且合适的值,但每个问题都可以使用这个初始值。由于向量中所有测试的值都是负数,所以没有一个大于错误初始化的值。这个错误引入了一个容器中未包含的外部值作为其最大值。

\mySamllsection{解决}

考虑在使用变量之前声明它——尽可能缩短声明、定义和使用之间的距离。在某些情况下(for 循环),可以在结构范围内声明变量,确保它仅在那里可见(现代 C++ 允许在 if 语句中这样做)。

初始化变量是一项有趣的练习。许多学生倾向于始终使用零值来初始化变量。在许多情况下,这是合适的,但并非全部。这部分将零视为错误的问题将导致微妙的问题。清单 5.2 中的 maximum 函数通过在一组负值中搜索最大值来演示这一点。出于某种原因,太多开发人员在编写代码时只考虑正值。整数和实数有一个令人讨厌的习惯,就是经常为负数,不容忽视。在扫描值集合之前,一种简单的初始化方法是将第一个元素的值复制到变量并比较其他值。这种方法可确保使用实际数据而不是开发人员假设的值初始化变量。如前所述,使用开发人员选择的值初始化可能会使用数据集中不存在的值,从而导致正确性错误。

\filename{清单5.2 在需要时声明变量}

\begin{cpp}
int maximum(const std::vector<int>& values) {
  // assume code here...
  int max = values[0]; // 1
  for (int pos = 1; pos < values.size(); ++pos) // 2
    if (values[pos] > max)
      max = values[pos];
  return max;
}

int main() {
  std::vector<int> values;
  values.push_back(1);
  values.push_back(-2);
  values.push_back(-3);
  std::cout << maximum(values) << '\n';
  return 0;
}
\end{cpp}

{\footnotesize
注释1:使用集合值之一初始化变量

注释2:将循环控制变量的范围限制在循环内
}

记住变量的含义和值对于代码的可读性和推理至关重要。在需要的地方准确声明变量并在使用前初始化它们可以减轻读者的认知负担。

\mySamllsection{建议}

\begin{itemize}
\item
限制每个变量的范围;声明后立即用有意义的值初始化它。

\item
使用在 for 循环范围和允许这种方法的任何其他构造中声明变量 的能力(现代 C++ 增加了更多机会)。
\end{itemize}












