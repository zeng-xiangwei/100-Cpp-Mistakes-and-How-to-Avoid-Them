
本书讨论每个错误相互独立,但在实际开发中,某些错误可能会对其他代码产生影响。这种关联性增加了诊断和解决问题的复杂性。为了便于理解,本书中的示例简化了这些复杂性,从而缩小了错误的影响范围。大多数章节专注于介绍特定错误的概念,但有些错误由于其广泛的适用性会出现在多个章节中。这表明 C++ 的许多特性之间存在交叉关系,单一特性和错误往往不仅限于影响语言的某一部分。每个错误分析都基于开发和执行中的四个关键特征展开。

\mySubsubsection{1.2.1.}{正确性}

衡量代码是否无误的解决了预期问题。如果代码是正确的,我们就可以确信问题已被妥善解决。尽管优雅且学术化的代码可能更具吸引力,但它们首先必须保证正确性。经验法则是:编写代码时应始终以解决问题为目标。虽然在实际开发中,完全正确的代码并不总是能够实现,但这仍然是一个值得追求的理想目标。

需要注意的是,在某些情况下,错误代码可能会引发未定义行为(UB)。未定义行为是一种极其危险的现象,应不惜一切代价避免(在此播放 Shaft\footnote{译者注:电影《杀戮战警》} 的主题曲)。UB 指的是语言标准未明确定义的行为,可能导致不可预测的结果、程序崩溃或安全漏洞。编译器可以以任意方式处理 UB,甚至对其进行优化。

\mySubsubsection{1.2.2.}{可读性}

衡量开发人员如何通过代码向其他开发者清晰传达意图的能力。每位开发者都有自己的风格,但如果这种风格妨碍了代码的清晰性和简洁性,则需要进行改进。建议参考通用的或团队内部的风格指南。经验法则是:编写代码时应始终以其他开发者为受众,而非编译器。通过表达自己的思路,使他人能够快速、清楚地理解代码逻辑,并严格遵守强制性的编程风格规范。尽管老练的开发者倾向于编写优雅的代码,但过于复杂的实现可能会给经验不足的开发者带来阅读困难。

\mySubsubsection{1.2.3.}{有效性}

衡量开发人员是否充分利用了语言特性来节省开发时间。紧凑的表达和对某些特性的熟练运用是有效性的核心。开发时间直接影响开发者的生产力,而生产力通常与代码的正确性相关。经验法则是:将正确的语言特性应用于适当的场景。

\mySubsubsection{1.2.4.}{能效性}

能效性衡量的是开发者选择最优方法、算法或数据结构以确保代码与硬件高效协作的能力。这是四个特征中最具挑战性的一个,因为直觉往往需要经过验证和调整才能实现最佳性能。建议使用分析工具定位实际的瓶颈和热点区域。

这一特征也是学术性最强的一个。选择合适的算法通常是提升效率的最佳途径,而这需要一定的计算机科学知识作为支撑。经验法则是:了解不同方法或算法的计算成本,并根据具体问题选择最合适的解决方案。
