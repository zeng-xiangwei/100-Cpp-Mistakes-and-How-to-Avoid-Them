
本书讨论的每个错误都是独立的。在现实世界中,一些错误会影响其他代码。这种相互关联使得诊断和解决更加困难。我们的方法消除了这种复杂性,但存在呈现错误更广泛影响的不完整视图的风险。大多数章节都介绍了所包含错误的总体概念。有些错误出现在多个章节中,因为它们的适用范围更广。这种情况揭示了许多 C++ 特性的交叉性质;一个特性或错误很少只影响该语言的一小部分。每个错误都根据涉及开发和执行的四个优先特征进行分析。

\mySubsubsection{1.2.1.}{正确性}

一个重要的错误类别是 正确性,它衡量代码是否解决了预期问题而没有错误。最终,代码可能以任何其他标准衡量都是一场灾难,但如果它是正确的,我们可以确信问题已经得到解决。最优雅、学术上最复杂的代码可能看起来很吸引人,但它必须是正确的。经验法则是编写代码来精确解决问题,同时不引入不必要的副作用。在许多情况下,现实情况并非如此干净,但这是一个值得追求的目标。

注意:在某些情况下,错误的代码可能会导致未定义行为 (UB)。UB 是一种极具冒犯性的粗俗行为,应学会不惜一切代价避免 UB(在此播放 Shaft 主题曲)。UB 是语言标准未定义的程序操作,会导致不 可预测的结果、崩溃或安全漏洞。编译器可以以 任何 方式处理 UB,包括对其进行优化。

\mySubsubsection{1.2.2.}{可读性}

第二个类别是 可读性 ,它衡量开发人员如何向其他开发人员传达他们对代码的意图。每个开发人员都有自己的风格,但如果它妨碍了 清晰、简洁的沟通,则应该改进。考虑普遍适用或面向团队的风格指南。经验法则是编写代码供其他开发人员使用(编译器不关心),表达自己,以便其他人清楚、快速地了解您的思维过程,并遵守任何强制性的风格指南。更老练的开发人员倾向于编写优雅的代码;这是自然的,但可能会给不太熟练的开发人员带来困难。

\mySubsubsection{1.2.3.}{有效性}

有效性衡量开发人员是否使用了为特定目的而设计的语言特性来节省开发时间。表达的紧凑性和某些特性的熟练使用是有效性的核心。解决问题所需的时间与程序员的生产力直接相关,通常与正确性间接相关。经验法则是将正确的特性用于其预期用途。

\mySubsubsection{1.2.4.}{性能表现}

最后,性能表现 是关于选择最佳方法、算法、数据结构或解决问题的方法,使其与硬件配合良好。这个方面是最难做到的,因为直觉往往需要纠正,才能让代码性能最佳。应该使用分析工具来发现实际的瓶颈和热点。

这一特征是四个因素中学术性最强的一个。为问题选择正确的算法通常是最好的首选方法,选择合适的算法需要一定的计算机科学知识。经验法则是了解各种方法或算法的计算成本,并选择最佳解决方案来有效解决问题。
