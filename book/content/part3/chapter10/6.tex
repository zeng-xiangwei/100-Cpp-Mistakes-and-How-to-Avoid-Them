此错误重点在于正确性和有效性。如果不理解,不熟悉该习语的读者可能会感受到负面影响。

当开发人员使用指针并且必须通过所有正常和错误路径管理动态和有限资源时,管理这些资源会很困难。在自动处理管理的类中实现的 RAII 模式是充分解决这些复杂性的一大步。但是,每次需要管理另一个资源时,都必须开发一个新类。

使用指针来管理动态资源是一种自 C 时代以来使用多年的模式,可以在 C++ 代码中找到。通过原始指针管理资源的困难促使人们提出了一种称为 auto\_ptr 的解决方案。这种类型的指针旨在管理原始指针的困难,但在一次罕见的重大失误中,它造成了重大问题,而且很奇怪。

\mySamllsection{问题}

许多遗留代码中都存在多个指向资源的原始指针,这些指针遍布整个代码库。之前的错误已经讨论了使用它们进行资源管理的一些缺陷。有人建议使用 RAII 模式,但必须开发新的类来管理每种资源类型。考虑到遵循该模式的开销,摆脱原始指针似乎是一件令人望而生畏的事情。

清单 10.10 中的代码是通过原始指针管理的一些资源的简化视图。我们假设进行更改很棘手,并且通常不被上层技术资源和管理层所接受。除了研究代码并模拟正常和错误路径之外,开发人员几乎没有其他选择。代码是正确的,但开发人员花费了太多时间思考和编写代码来管理 Student 实例。不幸的是,这一次很有价值,但目的不是解决问题。

\filename{清单10.10 使用原始指针管理资源}

\begin{cpp}
struct Student {
  std::string name;
  double gpa;
  Student(const std::string& name, double gpa) : name(name), gpa(0) {
    if (gpa < 0)
      throw std::invalid_argument("gpa is negative");
  }
};
int main() {
  Student* sammy = NULL; // 1
  Student* ginny = NULL;
  Student* gene = NULL;
  try {
    sammy = new Student("Samuel", 3.75); // 2
    ginny = new Student("Virginia", 3.8);
    gene = new Student("Eugene", -1);
  } catch (const std::invalid_argument& ex) {
    std::cout << "exception caught\n";
    if (sammy) // 3
      delete sammy;
    if (ginny)
      delete ginny;
    if (gene)
      delete gene;
    std::exit(1);
  }
  if (sammy) // 4
    delete sammy;
  if (ginny)
    delete ginny;
  if (gene)
    delete gene;
  return 0;
}
\end{cpp}

{\footnotesize
注释1:正常路径和错误路径上都必须有一个指针

注释2:创建实例并分配给指针

注释3:错误路径清理,重复

注释4:正常路径清理,重复
}

\mySamllsection{分析}

虽然代码按设计正确运行,但相当一部分仅管理基于指针的资源。原始指针在大部分代码中随处可见,但可以使用 RAII 模式将其管理最小化。此代码设计为在抛出异常时退出;因此,正常路径和错误路径不 能共享清理代码。这一事实要求重复清理代码。可以编写一个包含清理的函数,并从两个地方调用它,但这并不能从根本上改变这种情况。

经典 C++ 提供了一个实现 RAII 的通用类,可用于基于指针的资源。auto\_ptr 模板实现 RAII 模式并管理指向资源的指针。动态资源是在 auto\_ptr 之外创建的,而不是在其构造函数中创建的。

auto\_ptr 构造函数独占指针,以防止出现上一个错误中提到的问题。清单 10.11 中的代码建议使用 auto\_ptr 类进行彻底修复;但是,此类存在几个重大问题。这些困难至少分为三个重要方面:

\begin{itemize}
\item
复制此类指针时,源指针将被清零。

\item
这些指针无法在标准模板库 (STL) 容器中正确使用。

\item
可以创建这些指针的数组,但永远无法正确删除。
\end{itemize}

首先,复制 auto\_ptr 实例时,目标指针和源指针会被改变。源指针被赋值为 NULL复制之后。这是不寻常的复制语义,旨在防止访问共享资源;但是,从类名来看,这种行为并不明显。

其次,STL 容器要求实体可复制和可分配。如果将 auto\_ptr 添加到容器中,则容器将拥有资源的独占所有权,并且源指针将为空(这是上述问题的结果)。托管指针不可复制或分配(它们似乎是“可转让的”)。

第三,数组是唯一的选择,因为 STL 容器不适用于智能(托管)指针的集合。这种方法无法正常工作,因为 auto\_ptr 析构函数使用的是delete 形式,而不是 delete[] 形式。虽然可以创建这样的数组,但无法正确删除它,从而导致资源泄漏。

清单 10.11 展示了一种使用 auto\_ptr 托管资源的简单方法,但它们的局限性非常大,一般来说,不应使用它们。C++11 标准弃用了 auto\_ptr,并将其从 C++17 标准中删除。这种快速的弃用和删除表明了这些托管指针的有害性质及其用户对它们的看法。

\filename{清单10.11 使用auto\_ptr对象管理资源}

\begin{cpp}
struct Student {
  std::string name;
  double gpa;
  Student(const std::string& name, double gpa) : name(name), gpa(0) {
    if (gpa < 0)
      throw std::invalid_argument("gpa is negative");
  }
};
int main() {
  try {
    std::auto_ptr<Student> sammy(new Student("Samuel", 3.75));
    std::auto_ptr<Student> ginny(new Student("Virginia", 3.8));
    std::auto_ptr<Student> gene(
    new Student( "Eugene", -1)); // 1
  } catch (const std::invalid_argument& ex) {
    std::cout << "exception caught\n";
  }
  return 0;
}
\end{cpp}

{\footnotesize
注释1:正常路径和错误路径在退出作用域时执行析构函数
}

这种简单的用法按预期工作,但它相当平淡无奇。好处完全在于当指针超出范围时(无论是正常情况还是由于异常),对指针的管理。振作起来;还有更好的方法。

\mySamllsection{解决}

理想情况下,需要使用 RAII 来管理基于指针的资源。这种方法使开发人员在编写几个特殊类来处理这些资源时的工作变得复杂。通常,设计一个专门处理这种情况的类会更好。C++ 提供了 auto\_ptr 类来实现这个目的。然而,事实证明这是一个重大错误。此解决方案在现代C++ 中被弃用并删除,表明其使用始终存在问题。现代 C++ 为那些有幸能够使用它们的人提供了强大而有意义的智能指针。

这种情况并不意味着开发人员必须编写自定义 RAII 类来管理指针。Boost 项目中的一些坚定的开发人员已经为我们完成了这项工作(\url{https://www.boost.org/})。如果您不能使用现代 C++,那么使用 Boost 库( 即使只是为了智能指针)也是值得的。清单 10.12 中的代码演示了 boost::scoped\_ptr 的用法,其语义是独占所有权;对于共享语义,请使用 boost::shared\_ptr。这些Boost 指针按预期工作,并解决了劣质 std::auto\_ptr 的缺陷。这些 Boost 指针是现代 C++ std::unique\_ptr 和 std::shared\_ptr 的基础。请放心使用它们!

\filename{清单10.12 使用auto\_ptr对象管理资源}

\begin{cpp}
#include <boost/scoped_ptr.hpp>

struct Student {
  std::string name;
  double gpa;
  Student(const std::string& name, double gpa) : name(name), gpa(0) {
    if (gpa < 0)
      throw std::invalid_argument("gpa is negative");
  }
};
int main() {
  try {
    boost::scoped_ptr<Student> sammy(new Student("Samuel", 3.75));
    boost::scoped_ptr<Student> ginny(new Student("Virginia", 3.8));
    boost::scoped_ptr<Student> gene(
    new Student( "Eugene", -1)); // 1
  } catch (const std::invalid_argument& ex) {
    std::cout << "exception caught\n";
  }
  return 0;
}
\end{cpp}

{\footnotesize
注释1:当范围退出时,正常路径和错误路径都会执行析构函数
}

开发人员现在只需花费最少的时间和精力来编写管理代码。其余的精力则用于解决他们正在处理的问题。最小化的代码更容易编写(即更 有效),也更容易理解(即更易读)。

boost::scoped\_ptr 实例是管理指向 Student 对象的指针的模板的实例化。构造函数通过获取资源的指针来取得资源的所有权。其语义是独占的;无法分配。如果需要分配,请使用共享指针版本。每当指针超出范围时,动态资源就会被删除。请注意,这是预期的行为。

现代 C++ 允许使用 std::move 函数传输 std::unique\_ptr 实例,该函数将源实例转换为可移动对象,然后通过复制赋值运算符将其传输到目标对象。此代码片段演示了该技术:

\begin{cpp}
std::unique_ptr<int> ptr1 = std::make_unique<int>(42);
...
std::unique_ptr<int> ptr2 = std::move(ptr1);
\end{cpp}

\mySamllsection{建议}

\begin{itemize}
\item
尽可能用现代 C++(或 Boost,如果没有现代)智能指针替换原始指针。

\item
尽可能用至少 Boost 智能指针替换 auto\_ptr 实例;现代 C++ 版本要好得多。

\item
不要将 auto\_ptr 用于 STL 容器;它们的复制语义是意外的。

\item
不要将 auto\_ptr 用于数组;它们的析构函数行为将导致未定义的行为。

\item
在可能的情况下,请考虑使用现代 C++ unique\_ptr 和 shared\_ptr 作为首选替代方案。
\end{itemize}






















