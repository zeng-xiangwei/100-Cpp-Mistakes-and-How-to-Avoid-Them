这个错误关注的是正确性和可读性。使用简单的模式来实现重要的语义概念,通过在所有条件下删除动态或有限的资源来保持正确性。

动态和有限资源具有所有权限制,管理它们具有挑战性。编译器无法通过检测问题来提供帮助。开发人员有责任将每个分配与其相应的释放配对。当抛出异常时,这种配对会很复杂。正确的设计必须考虑正常和错误路径。错误 76 详细讨论了这种情况并提供帮助。更好的是,有一个通用解决方案,可以让开发人员从管理代码和将动态资源管理推向代码中解脱出来。最通用的解决方案是,制作遵循RAII 模式的资源管理类。

\mySamllsection{问题}

考虑这样一种情况:需要一个动态数组来保存一组测试分数,而这些分数的大小在编译时是未知的。用户代码确定测试分数的数量,分配数组,将分数输入数组,并将该信息传递给构造函数。构造函数(不必要地)分配其数组,复制输入值,并计算平均值。开发人员知道负值可能是一个问题,进行了适当的测试,如果发现错误值,则会引发异常。

以下代码试图做正确的事情,但必须明确在一个函数中创建的数据应如何传输到另一个函数。这种尝试部分解决了问题,但很笨拙。

\filename{清单10.7 出现异常时的资源泄漏}

\begin{cpp}
class Grades {
private:
  int size;
  double* grades;
  double average;
public:
  Grades(int size, const double* grades) : size(size), grades(0),
         average(0) {
    if (size == 0)
      return;
    this->grades = new double[size]; // 1
    for (int i = 0; i < size; ++i) {
      if (grades[i] < 0) // 2
      throw std::invalid_argument("negative score");
      this->grades[i] = grades[i];
    }
    double sum = 0.0;
    for (int i = 0; i < size; ++i)
      sum += this->grades[i];
    average = sum/size;
  }
  ~Grades() { delete [] grades; }
  double getAverage() const { return average; }
};

int main() {
  int count;
  std::cout << "Enter number of items to average: ";
  std::cin >> count;
  double* scores = new double[count];
  for (int i = 0; i < count; ++i) {
    std::cout << "Enter score: ";
    std::cin >> scores[i];
  }

  try {
    Grades g(count, scores);
    std::cout << "Average score is " << g.getAverage() << '\n';
  } catch (const std::invalid_argument& ex) {
    std::cout << ex.what() << '\n';
  }
  delete [] scores;
  return 0;
}
\end{cpp}

{\footnotesize
注释1:分配动态资源

注释2:抛出前未能删除资源
}

\mySamllsection{分析}

如果发现负值,构造函数将抛出异常。grades资源不会被释放,从而导致资源泄漏。代码中至少有三个错误。首先,构造函数分配了一个动态数组,然后测试无效条件,抛出异常而不考虑释放资源。其次,构造函数正在计算平均值,这不是它的职责。发生了太多事情掩盖了代码的意图,使得查看资源问题更加困难。第三,没有必要分配动态数组。数组副本对于正确访问测试分数数据并非必须,有更好的方法来获取所有权。

勇敢的开发人员读了一篇关于 RAII 的文章,并决定更新代码以解决其建议。将输入值数组传递给实现 RAII 的 Grades 类,以使实例能够管理数据。第一次尝试如清单 10.8 所示,其中构造函数不再分配动态资源,而是得到了所有权。

有几篇文章谈到了这种方法,指出构造函数不必创建动态资源,但可以承担所有权。这是好的建议,也是正确的建议;然而,共享资源的问题仍然存在,资源的创建者必须将完全所有权传递给 RAII 类。以下代码显示了这种混乱。

\filename{清单10.8 带有共享资源的RAII}

\begin{cpp}
class Grades {
private:
  int size;
  double* grades;
public:
  Grades(int size, double* grades) : size(size),
  grades(grades) {} // 1
  ~Grades() { delete [] grades; } // 2
  double getAverage() const {
    if (size == 0)
      return 0.0;
    double sum = 0.0;
    for (int i = 0; i < size; ++i) {
      if (grades[i] < 0)
        throw std::invalid_argument("negative score");
      sum += grades[i];
    }
    return sum/size;
  }
};

int main() {
  int count;
  std::cout << "Enter number of items to average: ";
  std::cin >> count;
  double* scores = new double[count]; // 3
  for (int i = 0; i < count; ++i) {
    std::cout << "Enter score: ";
    std::cin >> scores[i];
  }
  try {
    Grades g(count, scores);
    std::cout << "Average score is " << g.getAverage() << '\n';
  } catch (const std::invalid_argument& ex) {
    std::cout << ex.what() << '\n';
    delete [] scores; // 4
  }
  return 0;
}
\end{cpp}

{\footnotesize
注释1:通过接收获取资源

注释2:在所有条件下删除获取的资源

注释3:创建原始资源

注释4:在错误条件下删除原始资源
}

这项努力值得称赞,但需要尊重独特资源的语义。如果调用代码希望将所有权传递给RAII 类,绝不能事后管理或访问该资源 — 必须依赖于类授予的访问权限。执行此代码表明在错误条件下会发生双重释放,当代码删除指针并随后再次删除同一指针时,就会发生双重释放,从而导致未定义的行为。当发生这种情况时,我的系统会崩溃,这提醒这里真的有一个问题。

遵循 RAII 模式的析构函数正确地删除了资源,但开发人员误以为错误情况是,让原始所有者执行显式删除的原因——因此,进行了双重删除。如果没有抛出异常,RAII 类将正确处理所有事情。这种情况必须测试正常和错误路径,以确保正确性。

\mySamllsection{解决}

RAII 背后的核心思想是构造函数应该分配动态资源。许多情况下,最好的方法是将其作为构造函数的唯一职责(除了其初始化列表中所做的事情之外)。某些情况下,这可能不切实际,因此应该修改方法,以便获取动态资源是构造函数的最后一项任务。这样,只有在获取资源(资源获取)时,初始化阶段(即初始化)才会完成,从而避免在出现异常时发生资源泄漏。如果构造函数成功完成,则将在正常和错误条件下调用析构函数。如果获取失败,则不会调用析构函数;这种方法是正确的,没有资源可以释放。这种全有或全无的方法是 RAII 模式的核心。

以下代码有一个主要警告:RAII 类不完整,没有解决复制构造函数和复制赋值操作符。此外,一个好的 RAII 类可能会实现指针访问操作符。所有此类操作都必须进行编码以避免灾难,使用默认操作符来完全实现 RAII 类。

\filename{清单10.9 具有独占所有权的RAII}

\begin{cpp}
class Grades {
private:
  int size;
  int next;
  double* grades;
public:
  Grades(int size) : size(size), next(0), grades(0) {
    this->grades = new double[size]; // 1
  }
  ~Grades() { delete [] grades; } // 2
  void addGrade(double grade) { grades[next++] = grade; }
    double getAverage() const {
    if (size == 0)
      return 0.0;
    double sum = 0.0;
    for (int i = 0; i < next; ++i) {
      if (grades[i] < 0)
        throw std::invalid_argument("negative score");
      sum += grades[i];
    }
    return sum/size;
  }
};

int main() {
  int count;
  std::cout << "Enter number of items to average: ";
  std::cin >> count;
  Grades g(count); // 3
  for (int i = 0; i < count; ++i) {
    std::cout << "Enter score: ";
    double grade;
    std::cin >> grade;
    g.addGrade(grade);
  }

  try { // 4
    std::cout << "Average score is " << g.getAverage() << '\n';
  } catch (const std::invalid_argument& ex) {
    std::cout << ex.what() << '\n';
  }
  return 0;
}
\end{cpp}

{\footnotesize
注释1:初始化只包括资源分配

注释2:只在构造函数成功时调用

注释3:由RAII类获取动态资源

注释4:正常或错误控制路径都调用析构函
}

动态和有限资源(例如,数据库连接句柄)在除最小程序之外的所有程序中都很常见。开发者会努力确保他们的代码正常运行,但很容易忽视程序与系统交互的问题。必须考虑这种交互以确保程序正确性。管理动态和有限资源可能很复杂,但 RAII 模式提供了一种将管理机制推入资源拥有类的方法。只要有可能,就应该将管理责任转移到仅用于管理资源的类中。RAII 模式确保所有成功获取的资源都会在正确的时间自动删除,而无需开发者添加逻辑。

我一直在努力完全理解短语 resource acquisition is initialization,尽管这个概念的实现很简单。有人建议改用语义更精确的短语scope-bound resource management (SBRM, 范围受限的资源管理)。我修改了这个短语,添加了一些解释性文字,使其更明确:资源获取是(此对象的目的) 初始化。我听说 StackOverflow 上有人使用了短语destruction is resource relinquishment(破坏就放弃资源),这句话非常切中要害。Phil Karlton 说过,计算机科学中只有两件难事:缓存失效和命名。所以,我们就继续吧。

\mySamllsection{建议}

\begin{itemize}
\item
使用 RAII 管理动态和有限的资源;该模式可确保自动释放所有获取的资源。

\item
限制 RAII 类构造函数仅尽可能获取资源;如果不能,则将获取作为最后一站。

\item
如果构造函数传递了资源,请确保调用代码永远不会再次访问该资源 — 是 RAII 类管理的独占资源。
\end{itemize}





