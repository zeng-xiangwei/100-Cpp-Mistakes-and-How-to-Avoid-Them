这个错误主要针对正确性、可读性和有效性。不必要的访问器打破了类和用户之间的隔离。这些不必要的访问器需要花时间编写,但真正维护起来却很少;其会带来额外的认知负担,而且没有任何优点。

在面向对象编程的时代,许多教科书和作者都提倡为每个私有实例变量提供访问器和修改器。流行的说法是,每个用户都应该能够访问实例内的每个状态。这种代码淡化了更重要的方法,使阅读和记住类变得更加困难。遵循这一建议的代码,充斥着返回实例变量值副本的访问器。这种方法打破了公共/私有封装,过度的暴露了实现细节。

\mySamllsection{问题}

这种方法在很多方面都存在语义问题。如果变量存在非必要访问器,则变量和类实现可能会危险地暴露给用户。使用简单的访问器和修改器,公共变量和私有变量之间没有区别。泄露状态变量的实现细节表明,用户代码应该负责理解和管理它们,这个缺陷直接违反了维护类不变性的规定。修改类实现时,正确性和有效性会受到影响,因为用户代码可能依赖于已公开类的行为方式。

访问器的另一个不好的做法是,使用实例变量来存储从其他实例变量派生的对象状态。假设 Circle 类中计算面积,如清单 8.1 所示。遗留代码示例定义了一个名为 area 的浮点变量,该变量在构造函数中初始化。编写了一个简单的访问器,来返回 area 变量的计算值。

\filename{清单8.1 带有计算值和简单访问器的类}

\begin{cpp}
class Circle {
private:
  const static double PI;
  double radius;
  double area;
public:
  Circle() : radius(1), area(PI) {}
  Circle(double radius) : radius(radius),
    area(radius*radius*PI) {} // 1
  double getArea() const { return area; } // 2
  double getRadius() const { return radius; }
};
const double Circle::PI = 3.1415927;

int main() {
  Circle c1(3);
  std::cout << c1.getArea() << '\n';
  return 0;
}
\end{cpp}

{\footnotesize
注释1:需要计算的值

注释2:一个简单的访问器,只访问计算值
}

非必需访问器会导致代码膨胀。实例由其构造函数初始化后,部分或大部分实例变量可能对用户而言没必要了解的。如上所述,这些访问器会让用能依赖于变量的实现细节,从而阻止简单的重新实现。另一个结果是用户可能承担理解和维护实例状态的责任。这种情况违反了类不变性,并将类的封装传播到类外。

\mySamllsection{分析}

正确编写的类仅显示必要的状态信息,而不一定以类中使用的形式显示。公共接口定义了用户如何访问该信息,但没有定义该信息的实现方式、是缓存还是直接计算,或其他实现细节。用户不得依赖实现细节,类编写者有责任防止这种情况发生。

Circle 类使用传入的半径信息构造一个实例。访问器为用户提供了对 radius 值,及其实现的无限制可视化。area 方法依赖于预先计算的实例变量;其访问器仅返回该值。

应避免随意编写 radius 访问器;如果用户需要此信息,则应提供。需要使用用户代码验证该值,以查看是否需要;如果是,则提供访问器。由于似乎只有圆的面积才有意义,因此计算该值很重要,而半径则不重要。但确实需要存在,并且具有所有假定的优点。在添加半径访问器之前,应确定用户要求。用户已经提供了半径;如果是必需的,用户应该记住它。

第二个问题是面积的自动计算。用户可能需要这些信息,因此预先计算是一个好主意。但作为一种习惯,预先计算应该推迟到第一次(或每次)使用时。如果类更复杂,则不确定是否需要该面积 — 类设计者应该按照意图工作,并完成工作。

\mySamllsection{解决}

应将非必要的访问器删除。用户可能需要这些值,但不应将其作为“前提”来使用。扫描代码库,可以看到现有代码中是否使用了访问器。如果没有,则将假设需要值全部删除。如果使用了访问器,则确定用户代码是否因使用它而超越了其职责。示例代码中,如果用户是需要绘制和排列圆形的图形程序,则半径将是必需的。如果用户正在计算圆形铺路石的重量,则半径将无关紧要;只需计算面积。

\filename{清单8.2 带有计算访问器的类}

\begin{cpp}
class Circle {
private:
  const static double PI;
  double radius;
public:
  Circle() : radius(1) {}
  Circle(double radius) : radius(radius) {}
  double getArea() const { return radius*radius*PI; } // 1
};
const double Circle::PI = 3.1415927;
\end{cpp}

{\footnotesize
注释1:仅在需要时计算值
}

开发人员在编写计算访问器时必须采取平衡的方法,考虑计算成本。计算操作通常成本低廉,可按需计算。假设计算成本高昂(例如,使用大量数据、访问数据库或在线通信),则应尽可能少地计算其结果。这种情况下,更好的方法是仅在第一次访问时计算结果,然后缓存结果。后续访问将快速访问缓存的值,从而减轻计算成本。

由于大多数访问器不会改变实例变量,最好将其标记为const。编译器将强制承诺不会修改任何值;此外,这也是一种意图的良好文档式方案。

\mySamllsection{建议}

\begin{itemize}
\item
不要编写访问器,除非对于用户使用必不可少;必不可少的访问器是必需的。

\item
不要维护可以计算其值的实例变量,除非这是计算量很大的操作。

\item
出于性能原因,应该缓存昂贵的计算结果。

\item
将访问器标记为 const 以传达其不变性。
\end{itemize}











