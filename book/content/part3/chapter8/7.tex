此错误主要针对正确性。其他特性不会受到此错误明显影响。

继承是指从现有类中定义新类,从而允许统一处理相关对象的集合。层次结构中的每个类都会向组合的派生类添加一个数据部分,派生类的对象是这些数据部分的组合。多态是将这些派生类视为基类,但行为与派生类相同的运行时能力。相应的类构造函数、析构函数和操作符,必须正确管理层次结构中每个类的各个数据部分。很容易忽视复制赋值操作符的正确性。

\mySamllsection{问题}

复合对象(派生类实例)的每个部分,都必须正确实现每个相关操作符。构造函数和析构函数经常在此上下文中提及,如果不引用基类构造函数,就很难编写构造函数。应更频繁地提及、理解此上下文中操作符的正确操作。

考虑以下Person和Student之间的继承层次结构。Student 派生自Person,因为学生是人,但学生又不仅仅是人。Student 类的开发人员会小心地实现代码来处理该类中的专业知识,但可能会忽略基类中的类似知识。基类中的实例变量是私有的,派生类无法访问;所以派生类无法充分复制这些值。

清单 8.13 展示了从一个实例到另一个实例的复制操作。这个例子有些牵强,但这种类型的操作经常发生在未命名的变量中,例如vector或数组元素,并且没有明显的警告标志,例如:将 Thelma 复制到 Louise,如下面的清单所示。

\filename{清单8.13 派生类复制数据,但忽略其基类}

\begin{cpp}
class Person {
private:
  std::string name;
public:
  Person(const std::string& name) : name(name) {}
  const std::string& getName() { return name; }
};

class Student : public Person {
private:
  double gpa;
public:
  Student(const std::string& name, double gpa) : Person(name), gpa(gpa) {}
  Student& operator=(const Student& o) {
    if (this == &o)
      return *this;
    gpa = o.gpa; // 1
    return *this;
  }
double getGpa() const { return gpa; }
};

int main() {
  Student thelma("Thelma", 3.85);
  Student louise("Louise", 3.75);
  louise = thelma;
  std::cout << thelma.getName() << ' ' << thelma.getGpa() << '\n';
  std::cout << louise.getName() << ' ' << louise.getGpa() << '\n';
}
\end{cpp}

{\footnotesize
注释1:Student 实例变量可正确复制,但 Person 实例变量则没有
}

该问题的结果从其输出中可以看出:

\begin{shell}
Thelma 3.85
Louise 3.85
\end{shell}

\mySamllsection{分析}

当完成创建 louise 对象时,它会分配给 thelma 对象。现在,我们有两个 thelma 值,由于对 louise 实例的赋值操作,每个值都与不同的变量名相关联。赋值不会导致编译或运行时问题,因此代码按预期工作。但请注意,louise 对象仍然保留其名称 Louise,但使用 Thelma 的 gpa — 对于学生来说,这可能是最好的选择。但对于我们的编程完整性而言,事实并非如此。

派生类定义了复制赋值操作符,以确保参数的 gpa 复制到自身,代码按预期工作。但基类中的 name 字段也需要正确复制,其在 Person 中为私有,因此Student无法复制——这很容易忽视。只有从代码中删除复制赋值操作符定义,复制操作才能按预期工作。默认的复制赋值操作符和我们的定义不同。

编译器编写的默认复制赋值操作符,确保在对派生类的数据字段进行成员级复制之前,调用基类复制赋值操作符。这个微小但关键的事实,对于正确操作至关重要。由于派生对象是所有基类和派生类的组合,当复制或赋值组合的一部分时,必须确保每个部分都执行了复制或赋值操作。

\mySamllsection{解决}

理解了这个问题,就可以轻松解决。清单 8.14 中更新的代码使赋值操作正确执行。复制赋值操作符确保首先调用基类版本,然后继续其复制行为。复制赋值操作符确保调用其基类复制赋值操作符,并将参数作为参数传递,以确保处理层次结构中所有更高级别的实例。对象构造顺序所确立的主题再次体现:基类始终先构造、复制或赋值,然后在派生类中执行相同的操作。

\filename{清单8.14 适当考虑其基类的派生类}

\begin{cpp}
class Person {
private:
  std::string name;
  public:
  Person(const std::string& name) : name(name) {}
  const std::string& getName() { return name; }
};

class Student : public Person {
private:
  double gpa;
public:
  Student(const std::string& name, double gpa) : Person(name), gpa(gpa) {}
  Student& operator=(const Student& o) {
    if (this == &o)
      return *this;
    Person::operator=(o); // 1
    gpa = o.gpa; // 2
    return *this;
  }
  double getGpa() const { return gpa; }
};

int main() {
  Student thelma("Thelma", 3.85);
  Student louise("Louise", 3.75);
  louise = thelma;
  std::cout << thelma.getName() << ' ' << thelma.getGpa() << '\n';
  std::cout << louise.getName() << ' ' << louise.getGpa() << '\n';
}
\end{cpp}

{\footnotesize
注释1:首先调用基类复制赋值操作符

注释2:随后复制派生类实例变量
}

\mySamllsection{建议}

\begin{itemize}
\item
如果在派生类中定义了任何操作符,请确保在执行数据移动操作之前,实现包括对等效基类操作符的调用。

\item
构造函数和操作符的顺序必须遵循严格的模式,即最基类先执行,派生类最后执行,而析构函数则反转此顺序。
\end{itemize}














