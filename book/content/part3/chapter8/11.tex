这个错误主要关注正确性、有效性和性能。在理解这个习语之前,开发人员可能会说这会对可读性产生负面影响。然而,一旦理解了这个习语,大多数人都会更喜欢这种方法。这个错误应该很明显,没有必要;不幸的是,即使出于最好的意图,这个问题仍然会发生。

类将具有两种类型的实例变量:原始(内置)和类实例。许多现代语言强制初始化所有变量,无论其位置或用途如何。C++ 并不总是确保初始化发生。在某些情况下,实例变量已声明但没有定义的初始值。内存中的每个字节都有一些状态,因此这些变量将具有未定义的值。因此,读取这些变量会导致未定义行为;先赋值然后读取就可以了。初始化的目的是确保状态由程序明确定义。

初始化原始类型的三种常用方法:使用文字值分配变量;输入分配给变量的值;从另一个已初始化的变量分配变量。这些方法确保变量已声明并初始化。开发人员必须确保在第一次使用每个局部变量 之前都满足这一点。

对于类,初始化是构造函数的职责;它的主要工作是初始化变量 ,建立类不变量。构造函数必须接收每个参数的参数值,并使用它们初始化实例变量或为它们分配默认值。如果不是在每种情况下都这样做,则可能导致未定义的行为。

\mySamllsection{问题}

许多问题都以某种方式与人有关。假设我们需要用 name 和 age 来建模 Person 类。这个简单类的以下实现编译后似乎运行良好,只是它使用了未定义的数据。除非手动检查人员的年龄,否则它不会使其错误显而易见。

\filename{清单8.21 带有未初始化内置实例变量的类}

\begin{cpp}
class Person {
private:
  std::string name; // 1
  int age; // 2
public:
  Person(const std::string& name) { this->name = name; } // 3
  void setAge(int age) { this->age = age; }
  friend std::ostream& operator<<(std::ostream&, const Person&);
};

std::ostream& operator<<(std::ostream& o, const Person& p) {
  o << p.name << " is " << p.age << " years old";
  return o;
}

int main() {
  Person joey("Joey");
  std::cout << joey << '\n';
  return 0;
}
\end{cpp}

{\footnotesize
注释1:一个实例变量,它是一个类实例

注释2: 一个实例变量,它是一个原始类型

注释3:一个看似无辜的构造函数
}

开发人员应该坚持在构造函数中提供年龄(忽略变量是错误的格式) 。目的是让客户端代码通过调用 setAge 方法初始化年龄数据,从而确保 age 实例变量有效。当程序员需要记住这样做时,可能会出现问题。在这种情况下,编译器无法警告未初始化问题,因此您需要自己找出问题,而这个问题很可能会在代码使用一段时间后出现。

有些编译器在调试模式下运行时会初始化变量,但在发布模式下不会。当然,这意味着“它在我的机器上可以运行”,但客户并不满意。

\mySamllsection{分析}

类中的两个实例变量属于两个类别。std::string 对象是类实例,如果未尝试将值传递给它的构造函数。它的默认初始化确保字符串为空,是一个合法的对象;但是,它不包含对程序员的问题有用的值。但是等等,名称怎么会是空字符串呢?构造函数主体将其初始化为参数值,该参数值不太可能是空字符串。确实如此,但这超出了实际操作的范围。

构造函数主体中的代码是赋值,而不是初始化。在此错误的介绍部分中,术语 initialization 的使用有些含糊,其中赋值用于初始化。

构造函数确保在调用构造函数主体之前初始化所有类实例变量。这意味着在进入 Person 构造函数主体之前会调用默认的 std::string 构造函数。然后,在构造函数主体中,赋值将参数值的副本绑定到实例 变量。这是低效的;变量被初始化并赋值。执行了两个操作,但实际上只需要一个操作。更糟糕的是,如果参数不引用调用者的参数值,则会进行另一次复制以初始化参数。哇!

age 实例变量是内置类型。构造函数在进入构造函数主体之前会为它做什么?什么也不做。(谢谢 C,你的遗产在这里非常有用。)age 实例变量将是恰好位于其所在内存中的随机位模式。允许访问该值,但其内容未知 - 其含义和用途未定义。这种情况可能会导致漫长而有趣的调试会话。

\mySamllsection{解决}

编写构造函数的正确方法是确保每个实例变量都明确初始化。为每个实例变量提供一个参数值,或在初始化列表中提供一个有意义的默认值。坚持这种方法将确保每次都初始化每个实例变量。

消除多余的类类型实例变量初始化的解决方案是使用参数值以初始化列表形式初始化实例变量。将构造函数主体中的任何代码用于赋值给实例变量的情况作为极少数例外。

\filename{清单8.22 一个坚持为每个实例变量提供参数的类}

\begin{cpp}
class Person {
private:
  std::string name;
  int age;
public:
  Person(const std::string& name, int age) :
  name(name), age(age) {} // 1
  friend std::ostream& operator<<(std::ostream&, const Person&);
};

std::ostream& operator<<(std::ostream& o, const Person& p) {
  o << p.name << " is " << p.age << " years old";
  return o;
}

int main() {
  Person joey("Joey", 27); // 2
  std::cout << joey << '\n';
  return 0;
}
\end{cpp}

{\footnotesize
注释1:每个实例变量都必须初始化

注释2:每个构造函数参数都必须有一个实参
}

如果需要验证(例如变量),请编写一个私有验证方法,该方法返回已验证的参数值或引发异常。在初始化列表中调用验证方法。以下清单中的代码已更新,以包含 age 实例变量的私有验证器,该验证器在构造函数的初始化列表中调用。

\filename{清单8.23 在初始化列表中使用私有验证器验证参数}

\begin{cpp}
class Person {
private:
  std::string name;
  int age;
  static int validateAge(int age) { // 1
    if (age < 0)
      throw std::invalid_argument("negative age");
    return age;
  }
public:
  Person(const std::string& name, int age) :
  name(name), age(validateAge(age)) {} // 2
  friend std::ostream& operator<<(std::ostream&, const Person&);
};

std::ostream& operator<<(std::ostream& o, const Person& p) {
  o << p.name << " is " << p.age << " years old";
  return o;
}

int main() {
  Person joey("Joey", 27); // 3
  std::cout << joey << '\n';
  return 0;
}
\end{cpp}

{\footnotesize
注释1:返回有效值或引发异常的私有验证器方法

注释2:验证器在初始化列表中调用;不需要多余的赋值

注释3:因验证而安全地创建实例
}

\mySamllsection{建议}

\begin{itemize}
\item
确保每个实例变量在每个构造函数中都已初始化,无论是通过参数值还是默认值。

\item
请记住,内置类型不会在构造函数主体之外隐式初始化。

\item
不要依赖程序员记住初始化实例变量;在构造函数中提供参数,以便编译器提醒他们提供每个所需的参数。

\item
调用参数验证代码以确保正确建立类不变量;将验证器设为私有,在其中本地化实例变量的知识,并返回有效值或引发异常。
\end{itemize}
