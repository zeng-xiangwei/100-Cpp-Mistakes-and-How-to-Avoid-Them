这个错误主要针对正确性。在虚函数中误用构造函数和析构函数,会严重影响其他特性。

面向对象编程的显著优势是,能够创建相关类的层次结构。这些类通过扩展其基础类,使其具有特定行为,从而与其基础类相关联。这种能力源于根据定义特定行为的接口来处理相关类。专门的类会修改接口函数,从而为其特定类型产生有意义的结果。

这种特化允许对一般函数的行为进行细粒度控制。例如,Shape 基类可能具有 Circle 派生类。一般而言,所有形状都应该具有 area 方法,但如何实现计算完全取决于正在处理的形状类型。这种定义一般概念(在基类中)并用具体内容(在派生类中)覆盖该行为的想法称为运行时多态。多态性是面向对象编程的三大公认支柱。了解特定编译器如何实现多态性没有帮助;从概念上了解如何实现多态性,有助于理解为什么不合适在构造函数中调用虚函数。

考虑一下继承是一种自上而下的信息流。基类知道什么(状态,如实例变量)和做什么(行为,如方法)向下流向派生类。多态行为是自下而上搜索特定函数定义。从概念上讲,对适当虚函数的搜索从特定的派生类类型开始,并沿着层次结构向上移动。使用该函数的第一个定义版本。

编译器确定何时需要非虚拟实例方法的地址,并留出空间用于插入该方法的地址。当实例调用这些方法之一时,编译器或链接器会将地址插入可执行代码中。在运行时,该地址用于定位方法的代码。但对于虚函数,需要的不仅仅是这种方法。

虚函数的存在会导致编译器为该类建立一个新表,该表保存每个虚函数的地址在其中声明。这些地址不会像非虚函数那样在编译或链接期间直接插入到代码中。创建实例时,每个虚函数的指针用于引用正确的虚函数。实际细节很复杂,但以下概念模型是一个可行的想法。构造函数确定哪个类表代表正确的虚函数,并调整指针以引用。

构造顺序是精确定义的。假设一个三级层次结构:基类是 A,中间类是 B,最外层派生类是 C。当创建 C 的实例时,将调用其构造函数。

构造函数做的第一件事是调用 B 的相应基类构造函数。B 构造函数通过调用其基类构造函数来初始化 A 部分,从而开始执行。在 A 构造函数完成初始化 A 部分后,B 构造函数继续执行并完成初始化 B 部分。

\myGraphic{0.5}{content/part3/chapter8/images/1.png}{图 8.1 显示构造顺序的三级层次结构}

只有在 B 构造函数返回后,C 构造函数才会恢复执行并初始化对象的C 部分。图 8.1 显示了此构造顺序,其中构造函数首先调用其基类构造函数。

对于具有虚函数的类,认为构造函数负责将指针调整到正确的类表;这在技术上并不正确,但其解释了为什么派生类有最终决定权。由于A 构造函数首先运行完成,所以将使用定义虚函数的 A 类表初始化表。如果 B 也定义了虚函数,则 A 类的表指针将使用 B 类表的位置覆盖,C 类也是如此。因此,当通过基类指针处理 C 实例时,用于查找正确虚函数的表将是最近调整的表,即使 C如果实例被视为通用的 A 对象,则写入实例的最后一个表指针就是调用虚函数时调用的指针。当调用虚函数时,通用的 A 指针或引用可以表现得像 C 实例一样。

这种构造顺序的结果是,实际对象(本例中为 C 实例)将仅在其构造的某个时刻调整其类表指针。调整之前的时刻,对虚函数的调用会调用不正确且意外的版本。当构造函数或析构函数正在执行时,对象要么正在构造,要么正在销毁。由于对象仍然需要完全构造,不能认为已经准备好,并且不能保证类的不变性。

\mySamllsection{问题}

假设定义了一个继承层次结构,其中不同的人被赋予不同的尊称。在这所学校,研究生受到高度尊敬,学生受到一定程度的尊敬,而普通人则几乎不受到尊敬。由于这个层次结构需要表现得一般化,因此 GradStudent 和 Student 实例将根据需要替换为 Person 实例。但在这种情况下,这些替换的实例必须通过确定适当的尊称来具体表现。

清单 8.17 中的示例代码展示了一个案例,本来应该调用实际类型的getHonorific 虚函数并输出正确的信息。但是,由于这个错误,它失败了。

此示例显示了一个三级层次结构。每个构造函数都会初始化其实例变量,然后调用 print 方法。当此方法执行时,它会调用虚getHonorific 方法。输出显示每次调用 print 都会输出不同的消息。getHonorific 虚函数在每个部分的构造过程中将其函数指针添加到表中,这是 print 方法调用的特定函数。

\filename{清单8.17 从构造函数调用虚函数}

\begin{cpp}
class Person {
private:
  std::string name;
public:
  Person(const std::string& name) :
  name(name) { print(); }
  void print() const { std::cout << getHonorific() << name << '\n'; }
  const std::string getName() const { return getHonorific() + name; }
  virtual std::string getHonorific() const { return ""; }
};

class Student : public Person {
private:
  std::string year;
public:
  Student(const std::string& name, const std::string& year) : Person(name),
    year(year) { print(); }
  std::string getHonorific() const { return year + " "; }
};

class GradStudent : public Student {
private:
  bool candidate;
public:
  GradStudent(const std::string& name, const std::string& year,
    bool candidate) : Student(name, year), candidate(candidate)
  { print(); } // 1
  std::string getHonorific() const {
    return candidate ? "candidate " : "";
  }
};

int main() {
  GradStudent aimee("Aimee", "second year", true);
  return 0;
}
\end{cpp}

{\footnotesize
注释1:构造函数中调用虚函数;函数版本很可能不正确
}

\mySamllsection{分析}

代码的结果表明,构造每个部分后,将调用其版本的getHonorific,而不是实际类型的版本,实例尚未完全构造。因此,虚函数的版本和结果完全取决于对象的哪个部分正在调用该方法。以下输出显示了构造对象时的各种敬语,每个类类型一行:

\begin{shell}
Aimee
second year Aimee
candidate Aimee
\end{shell}

另一个问题是虚函数可能依赖于给定部分的状态数据。如果构造函数尚未完成该部分,则无法保证数据处于有效状态。如果调用虚函数并且它依赖于此不完整的数据,则结果可能不确定。

\mySamllsection{解决}

解决方案是在对象完全构造后调用虚函数。只有这样,虚函数表才会正确初始化,行为才会正确实现多态。以下代码演示了此修复。

\filename{清单8.18 没有虚函数调用的构造函数}

\begin{cpp}
class Person {
private:
  std::string name;
public:
  Person(const std::string& name) : name(name) {}
  void print() const { std::cout << getHonorific() << name << '\n'; }
  std::string getName() const { return getHonorific() + name; }
  virtual std::string getHonorific() const { return ""; }
};

class Student : public Person {
private:
  std::string year;
public:
  Student(const std::string& name, const std::string& year) : Person(name), year(year) {}
  std::string getHonorific() const { return year + " "; }
};

class GradStudent : public Student {
private:
  bool candidate;
public:
  GradStudent(const std::string& name, const std::string& year,
      bool candidate) :
    Student(name, year), candidate(candidate) {}
  std::string getHonorific() const { return candidate ? "candidate " : ""; }
};

int main() {
  GradStudent aimee("Aimee", "second year", true);
  aimee.print(); // 1
  return 0;
}
\end{cpp}

{\footnotesize
注释1:对象构造后调用虚函数
}

对于正确设计的类,析构函数的工作顺序与构造函数相反。当对指针或引用调用析构函数时,最先调用派生类的析构函数。其执行结束时,会调用其基类的析构函数。这会一直渗透到层次结构的顶部,从而保证对象以与构造相反的顺序销毁。

如果在析构过程中调用虚函数,则最外层派生类中的状态信息可能无效,具体取决于其析构函数的行为。但虚函数会期望这些数据处于有效状态,由于无法保证这一点,因此应始终避免在析构函数中使用虚函数。

\mySamllsection{建议}

\begin{itemize}
\item
确保在实例完全构造之前未调用类的虚函数;切勿从其构造函数中调用类的虚函数。

\item
构造顺序会影响状态信息的有效性,以及将调用哪个版本的虚函数;在调用任何虚函数之前,先让构造完成。

\item
销毁顺序会影响虚函数可用状态信息的有效性,其中一些信息可能会销毁或以其他方式无效;当销毁开始时,就永远不要调用虚函数。
\end{itemize}


