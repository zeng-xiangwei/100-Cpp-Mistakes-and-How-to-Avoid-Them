此错误重点在于正确性。正确和不正确的实现之间的差异可能不会影响可读性、有效性或性能(如果影响,则取决于编译器的实现)。

必须仔细考虑构造函数,因为正确初始化实例变量对于保持类不变至关重要。同样,必须仔细考虑析构函数。C++ 赋予开发人员类设计的权力和责任。必须考虑类以确定它是否应该使用继承、从继承而来或成为独立的数据类型。

\mySamllsection{问题}

独立类是最容易规划的。这并不意味着设计一定很容易;它只意味着没有继承,事情会更简单,而且你有更多的控制权。基类的设计会影响派生类的设计。在继承层次结构中犯下的错误可能会给派生类的开发人员带来痛苦。除了谨慎编程和考虑例外情况外,你可能对此无能为力。但是,如果你正在编写基类,你不仅要考虑你的类型,还要考虑所有的子类型——你的基类的行为将影响每一个派生类。

如果将某个类设计为基类,则它应遵循一些规则。其中一条规则是,基类应提供虚拟析构函数,因为它应至少具有一个虚拟函数。反过来说,如果没有任何虚拟函数,则不要提供虚拟析构函数。除非打算将类设计为基类,否则类不应包含虚拟方法;因此,每个基类都应提供虚拟析构函数。

使用基类指针或引用删除派生对象时,C++ 会让我们感到惊讶。多态行为只有使用指针或引用才有可能,因此必须使用这种方法。指针和引用并不意味着基于堆的对象,但这是一种非常常见的用例。那么问题是什么?尝试使用基类指针或引用删除派生类。似乎一切都应该没问题,但结果谁也说不准;更糟糕的是,这是未定义的行为。

清单 8.15 中的代码有一个基类和一个使用动态资源的派生类。基类的析构函数未定义为虚拟的。主函数创建一个动态 Square 对象并将其赋值给 Shape 指针。使用完派生对象后,将其删除。

\filename{清单8.15 没有虚析构函数的基类}

\begin{cpp}
class Point {
private:
  double x;
  double y;
public:
  Point(double x, double y) : x(x), y(y) {}
};

class Shape {
private:
  Point* location;
public:
  Shape(double x, double y) : location(new Point(x, y)) {}
  ~Shape() { delete location; std::cout << "~Shape\n"; } // 1
  virtual double area() const = 0; // 2
};

class Square : public Shape {
private:
  double* side;
public:
  Square(double side, double x, double y) : Shape(x, y),
  side(new double(side)) {}
  ~Square() { delete side; std::cout << "~Square\n"; }
  double area() const { return *side * *side; } // looks weird
};

int main() {
  Shape* s = new Square(2.5, 0, 0); // place at origin
  std::cout << s->area() << '\n';
  delete s; // 3
  return 0;
}
\end{cpp}

{\footnotesize
注释1:非虚拟析构函数,这似乎是正确的

注释2:用于多态行为的虚拟方法

注释3:删除对象的基类部分,但不删除派生类部分
}

此代码的输出为

\begin{shell}
6.25
~Shape
\end{shell}

这表明基类析构函数运行但泄漏了动态 side 值——您的结果可能会有所不同。

\mySamllsection{分析}

main 函数的输出显示,只调用了 Shape 析构函数。尽管这两个类的设计很不合理,但 Square 析构函数未被调用这一事实表明存在内存泄漏。

基类应该至少有一个虚函数,因为只有虚函数才能参与多态行为。否则,继承可能被用于代码重用。没有虚拟析构函数的基类可能看起来设计得当,因为它可以工作。当通过基类指针操作派生对象时,一切都按预期工作。也就是说,它一直工作到对象被销毁。行为变得奇怪,但不太可能被注意到。这种情况完全取决于编译器如何实现此功能;调试和发布版本也可能不同。

通过基类指针销毁派生类可能只会删除定义基类的部分。基类的析构函数将被调用并执行其任务。基类不知道(也不应该知道)应该删除对象的派生类部分。最终结果可能是派生部分成为内存泄漏。这种情况无疑是一个问题,但可能会避免短期运行的程序出现重大问题 - 这是欺骗行为。但是,当派生类具有动态资源(例如数据库连接、文件句柄或其他唯一或受限资源)时,问题更有可能影响操作。

\mySamllsection{解决}

解决方法很简单:始终为具有一个或多个虚函数的基类提供虚析构函数。此外,不要使用没有虚方法的基类或具有虚方法但不是基类的类。构造顺序始终是从继承层次结构中最基类向下到最派生类。相反,销毁顺序是从层次结构中最派生类向上到最基类 — 与构造相反。必须首先构造基类,因为派生类可能依赖于基类知识。这意味着应该首先销毁最派生类。除非使用 virtual 关键字,否则不会发生这种情况。

\filename{清单8.16 带有虚析构函数的基类}

\begin{cpp}
class Point {
private:
  double x;
  double y;
public:
  Point(double x, double y) : x(x), y(y) {}
};

class Shape {
private:
  Point* location;
public:
  Shape(double x, double y) : location(new Point(x, y)) {}
  virtual ~Shape() { delete location; // 1
  std::cout << "~Shape\n"; }
  virtual double area() const = 0;
};

class Square : public Shape {
private:
  double* side;
public:
  Square(double side, double x, double y) : Shape(x, y),
    side(new double(side)) {}
  ~Square() { delete side; std::cout << "~Square\n"; }
  double area() const { return *side * *side; } // still looks weird
};

int main() {
  Shape* s = new Square(2.5, 0, 0); // place at origin
  std::cout << s->area() << '\n';
  delete s;
  return 0;
}
\end{cpp}

{\footnotesize
注释1:添加虚拟关键字;现在,首先调用派生类的析构函数
}

此执行的输出显示预期的派生类销毁,随后是基类销毁:

\begin{shell}
6.25
~Square
~Shape
\end{shell}

基类析构函数中添加了 virtual 关键字,结果显示两个析构函数均被调用。这一令人欣慰的事实表明,两个类中的动态分配内存均被正确删除。如果基类和派生类使用任何动态资源(例如堆内存、系统资源句柄或类似实体),请确保它们具有析构函数。

\mySamllsection{建议}

\begin{itemize}
\item
多态行为需要指针或引用和虚拟函数;值对象(即基类元素)将分割派生部分。

\item
如果某个类具有虚拟函数,则该类应为基类并从中派生;始终遵循此模式。

\item
抵制从没有虚拟析构函数的现有类派生类的诱惑(或修复该现有类以进行继承)。

\item
如果类具有任何虚函数,请始终定义一个虚析构函数,以防止仅破坏基类部分,造成内存泄漏。

\item
如果类没有虚函数,则不要提供虚析构函数。这遵循了避免使用不打算继承的虚方法的类的建议。
\end{itemize}








