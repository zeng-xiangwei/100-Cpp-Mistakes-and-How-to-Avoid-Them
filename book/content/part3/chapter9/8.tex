这个错误主要影响性能;可读性和有效性可能会受到轻微的负面影响。

C++ 编译器中使用混合模式计算和函数调用非常容易,这种情况通常发生在将函数调用,或算术运算应用于两种不同类型时。如果编译器可以找到一种方法,将值从一种类型转换为另一种类型。从而使调用成功,那么它将默默地(通常)完成此操作,但这种转换可能会有性能损失。

\mySamllsection{问题}

隐式转换感觉很好,当它们顺利且正确地工作时很有趣。清单 9.20 中的代码显示了一个适度的 Complex 类,考虑到有人可能希望将双精度值添加到实数部分。不明显的是,没有办法进行这种简单的添加。定义的 operator+ 接受两个 Complex 参数;显然,double 不是Complex——需要静默的、可能有用的构造函数类型转换来救援!

\filename{清单9.20 隐式、静默和昂贵的构造函数类型转换}

\begin{cpp}
class Complex {
private:
  double real;
  double imag;
public:
  Complex(double real, double imag=0) : real(real), imag(imag) {}
  double getReal() const { return real; }
  double getImag() const { return imag; }
};
const Complex operator+(const Complex& lhs, const Complex& rhs) {
  return Complex(lhs.getReal()+rhs.getReal(), lhs.getImag()+rhs.getImag());
}

int main() {
  Complex c1(2.2); // 1
  Complex c2 = c1 + 3.14159; // 2
  Complex c3 = 2.71828 + c1; // 2
  Complex c4 = 2.71828 + 3.14159; // 1
}
\end{cpp}

{\footnotesize
注释1:一次构造函数调用

注释2:两次构造函数调用
}

如预期的那样,调用 Complex 构造函数来创建对象 c1,还为对象 c2 和 c3 调用了两次。第一次调用是转换调用,其中 double 值转换为等效的 Complex 表示。double 值初始化 real 组件,而 imag 组件默认为零。最后,对象 c4 是单个构造函数调用。4行代码中有6次构造函数调用。如果这是一个更复杂的数据类型,还会调用6次析构函数。在这种简单的纯数据类型的情况下,大多数(如果不是全部)编译器都会优化析构函数。

\mySamllsection{分析}

Complex 对象 c1 和 c4 的构造显而易见,每个对象只需调用一次构造函数。这种行为无法消除。对象 c2 和 c3 必须先将 double 转换为Complex 对象,然后执行加法,因为 operator+ 只接受 Complex 对象。转换需要调用一次构造函数;加法的结果需要调用第二次构造函数。将返回的对象赋值给声明的对象不需要构造函数调用,RVO 会就地复制。

此操作的所有方面都是正确的,但如果性能是关注点,构造函数(和析构函数)调用可能会成为痛点。如果不关心性能,这仍然是一种低效的方法。我们需要一种不需要调用构造函数,即可进行混合模式转换的方法。

\mySamllsection{解决}

性能不是问题的情况下,上述方法可以接受,并且代码量最少。这提高了可读性和效率(小菜一碟)。但在不允许调用构造函数(和析构函数)的情况下,必须有一种方法可以最大限度地减少这些调用。

解决方案是使用不同的参数列表重载 operator+,这些参数列表可以接受 Complex 和 double 的组合,但需要多写几行代码(效率会降低)。两个参数均为 double 类型的情况不包括在内,也不可能包括。C++ 确保重载操作符至少具有一个用户定义的参数类型。

如果编译器允许这种情况,则将重新定义两个 double 值的相加,从而导致与内置的相加规则不一致的行为。他们考虑到了一切!性能问题的解决方案很简单:为预期使用的每个组合编写一个重载的operator+。以下代码显示了 Complex 类的改进。再添加两个操作符,即可完成可能的重载参数列表集(Complex/Complex、double/Complex 和 Complex/double)。

\filename{清单9.21 通过提供重载操作符来最小化构造函数转换}

\begin{cpp}
class Complex {
private:
  double real;
  double imag;
public:
  Complex(double real, double imag=0) : real(real), imag(imag) {}
  double getReal() const { return real; }
  double getImag() const { return imag; }
};

const Complex operator+(const Complex& lhs, const Complex& rhs) {
  return Complex(lhs.getReal()+rhs.getReal(), lhs.getImag()+rhs.getImag());
}

const Complex operator+(const Complex& lhs, double rhs) { // 1
  return Complex(lhs.getReal()+rhs, lhs.getImag());
}

const Complex operator+(double lhs, const Complex& rhs) { // 1
  return Complex(lhs+rhs.getReal(), rhs.getImag());
}

int main() {
  Complex c1(2.2);
  Complex c2 = c1 + 3.14159;
  Complex c3 = 2.71828 + c1;
  Complex c4 = 2.71828 + 3.14159;
}
\end{cpp}

{\footnotesize
注释1:重载操作符定义消除了混合模式的构造函数转换
}

确保操作符具有所有必要的重载,这将最大限度地减少临时对象的影响。构造函数(和析构函数)调用的数量已减少到4个。此代码表明,在实现混合模式计算、函数调用和类型转换构造函数时,性能会受到显著影响。但不要过度使用 - 只重载必要的操作符,而不是所有操作符。困惑吗?只是尽力的平衡而已。

\mySamllsection{建议}

\begin{itemize}
\item
如果担心性能,则对混合模式函数或操作符调用重载每个预期模式;如果性能不是问题,这仍然是一种很好的彻底方法。

\item
操作符必须至少具有一种用户定义的参数类型,以保持语义一致性并避免重新定义现有规则。
\end{itemize}
