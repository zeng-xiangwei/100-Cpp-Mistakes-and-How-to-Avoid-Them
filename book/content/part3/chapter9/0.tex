本章内容

\begin{itemize}
\item
对比操作符和成员函数

\item
处理自赋值和返回值优化

\item
与类成员的编译器协作

\item
对比显式和隐式转换

\item
增量和减量操作符的前缀和后缀版本
\end{itemize}

本章继续讨论如何更好地使用实例。类不变量的概念始终存在,但解决这些错误需要与前几章有所不同。这当然并不意味着它不那么重要,只是重点更广泛,涉及不一定直接影响对象状态的领域。

这些错误经常涉及性能类别,其中一些错误通过消除不必要的临时对象来强调这一方面。当表达式求值的中间步骤需要中间对象来保存部分求值的结果(这些结果将在进一步求值中使用)时,就会出现这些临时对象。了解这些临时对象的创建时间,以及如何设计一个类来消除其中的许多临时对象,会显著影响构造函数和析构函数调用的次数。

其他错误集中在误用影响常见用法或性能的操作符。C++ 提供了大量为开发人员提供灵活性,并且必须尊重这种灵活性,以避免不当使用功能。
