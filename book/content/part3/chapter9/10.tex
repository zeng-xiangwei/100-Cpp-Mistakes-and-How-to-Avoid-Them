这个错误关注正确性,并稍微提高了效率。当执行类实例方法时,代码主体可以访问实例变量,并在需要时修改它们。这种行为是好的和正常的,但并非所有方法都需要修改变量。const 方法永远不会修改实例变量,只直接或间接返回这些状态变量。直接方法称为访问器,间接方法称为 计算访问器。

\mySamllsection{问题}

假设有一个相当大的类,并且它的方法太大,难以阅读和理解。还要考虑一些访问器类型的方法中嵌入了逻辑,这些方法可以很容易地(也许是错误的)修改状态变量。编译器无法防止对状态的意外修改。当这种情况发生时,对象可能会破坏类的不变量;如果是无意的,对象最终会处于不正确的状态。以下代码是此问题的一个简化版本,消除了所有模糊代码,并暴露了基本框架。

\filename{清单9.25 无意中修改状态的复杂访问器方法}

\begin{cpp}
class Person {
private:
  std::string name;
  int age;
public:
  Person(const std::string& name, int age) : name(name), age(age) {}
  const std::string& getName() { return name; }
  int getAge() { // 1
    ++age; // oops, unintentional
    return age;
  }
};

int main() {
  Person amy("Aimee", 26);
  std::cout << amy.getName() << " is " << amy.getAge() << " years old\n";
  return 0;
}
\end{cpp}

{\footnotesize
注释1:也许这个方法应该改变状态
}

\mySamllsection{分析}

虽然清单 9.25 中的代码非常简单,错误也很明显,但更复杂的代码会通过引入多行代码来掩盖状态修改,其中一些代码很难理解。这种混乱的代码中,很容易无意中修改状态。如上所述,编译器无法发出警告,其无法知道修改是正确的还是错误的——它必须假设更改是正确的。需要的是一些方法来避免无意修改变量的状态。

\mySamllsection{解决}

使用 const 关键字可以在很大程度上避免无意中修改状态变量。虽然没有关键字可以保证代码正确,但 const 关键字至少可以确保,实例变量不会在某些方法中修改。当方法用 const 关键字标记时,编译器会确保该方法不会修改实例变量。

建议将每个可以这样做的方法标记为 const,错误地将方法标记为const 也无妨。经检查,可以确定该方法不应使用该关键字,可以删除。对于所有其他方法,即那些有意改变状态的方法,该关键字是不必要的(实际上,这将是一个错误)并且不会使用。

编译器无法确保修改是必要的,但可以保证强制执行不修改。清单9.26 中的代码修改了 getAge 方法并发出错误消息,这是由于无意修改了代码。分析代码后,确定修改是一个错误(可能是错别字) 并消除。由于该方法不应修改状态,因此 const 关键字确保编译器不允许实例数据的更改。

\filename{清单9.26 将方法标记为static以避免修改状态}

\begin{cpp}
class Person {
private:
  std::string name;
  int age;
public:
  Person(const std::string& name, int age) : name(name), age(age) {}
  const std::string& getName() const { return name; }
  int getAge() const { // 1
    return age;
  }
};
int main() {
  Person amy("Aimee", 26);
  std::cout << amy.getName() << " is " << amy.getAge() << " years old\n";
  return 0;
}
\end{cpp}

{\footnotesize
注释1:const 关键字可避免实例变量发生改变
}

\mySamllsection{建议}

\begin{itemize}
\item
将所有非变异方法标记为 const,以防止无意中修改状态;成本很小,但可以保证的安全性。
\end{itemize}












