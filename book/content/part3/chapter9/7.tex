这个错误主要集中在可读性上,其次是正确性。阅读代码应该能说明全部情况;隐式转换会隐藏一些细节。

当函数调用未收到预期类型的参数时,C++ 提供了一组复杂的规则,用于将一种数据类型转换为另一种数据类型。使用两种形式的隐式转换,具体取决于是否定义了转换函数和转换构造函数。此错误主要针对转换函数。

\mySamllsection{问题}

转换操作符使用 operator 关键字定义,后跟要转换的数据的类型,后跟括号。例如,清单 9.18 中的 Rational 类可能返回表示有理数近似值的 double 值。double() 操作符是 Rational 类成员,将Rational 值转换为其近似值 double。

\filename{清单9.18 带有转换函数的类}

\begin{cpp}
class Rational {
private:
  int num;
  int den;
public:
  Rational(int num, int den = 1) : num(num), den(den) {}
  operator double() { return (double)num/den; }
  friend std::ostream& operator<<(std::ostream&, const Rational&);
};

std::ostream& operator<<(std::ostream& out, const Rational& r) {
  out << r.num << '/' << r.den;
  return out;
}

int main() {
  Rational r1(3);
  std::cout << r1 << ' ' << (double)r1 << '\n'; \\ 1
  if (r1 == 3) \\ 2
    std::cout << "equal\n";
  else
    std::cout << "not equal\n";
  return 0;
}
\end{cpp}

{\footnotesize
注释1:显式转换非常易读

注释2:隐式转换容易引起误解;看起来像是在比较整数
}

清单 9.18 中的代码按预期运行,因为定义了友元函数 operator<{}< 。其输出为

\begin{shell}
3/1 3
equal
\end{shell}

没有办法准确预测要比较的实际值(0.333 值)。我的学生经常听到我说:“浮点值是近似值,很少精确。”这种比较浮点值的方法是错误的,应始终使用 delta-epsilon(无穷小分析) 方法进行比较。

\mySamllsection{解决}

如果需要转换函数,最好使用函数而不是操作符。当编译器隐式尝试将值从一种类型转换为另一种类型时,不会考虑该函数。在无法进行 转换的情况下,编译器将出现错误,从而提醒开发人员函数不足。开发人员可能会意识到所编码的内容没有意义,并选择不同的方法。

下面的代码通过添加转换函数,并消除隐式转换来避免这两个问题。此外,对读者来说,更明显的是发生了显式转换。

\filename{清单9.19 使用显式转换函数的类}

\begin{cpp}
class Rational {
private:
  int num;
  int den;
public:
  Rational(int num, int den = 1) : num(num), den(den) {}
  double toDouble() { return (double)num/den; }
  friend std::ostream& operator<<(std::ostream&, const Rational&);
};
std::ostream& operator<<(std::ostream& out, const Rational& r) {
  out << r.num << '/' << r.den;
  return out;
}
int main() {
  Rational r1(3);
  std::cout << r1 << ' ' << r1.toDouble() << '\n';
  if (r1.toDouble() == 3)
    std::cout << "equal\n";
  else
    std::cout << "not equal\n";
  return 0;
}
\end{cpp}

删除 double 操作符并添加 toDouble 函数使隐式变得显式。现在,比较对读者来说更加明显,并且类型不匹配应该显而易见。由于比较双精度值,而非有理数,所以存在不精确性。使用 operator== 比较双精度是一个错误的选择。

\mySamllsection{建议}

\begin{itemize}
\item
尽量少用隐式类型转换操作符,除非读者能明显看出它们的用途,但这种情况很少见。

\item
编写显式类型转换函数可增强可读性,并避免意外转换和错误假设。
\end{itemize}
