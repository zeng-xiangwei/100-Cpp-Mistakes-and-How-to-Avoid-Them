
一个类至少有四个基本特征。一些学者可能会提出其他建议,但应该将这四个视为作为基本:

\begin{itemize}
\item
解释

\item
操作集

\item
值范围

\item
内存机制
\end{itemize}

\mySubsubsection{7.1.1.}{解释}

类不变量的主要特征是对类数据的解释。类是数据和函数的集合。数据是状态,函数是类实例的行为。在此上下文中,instance 用于表示从程序的角度来看类的创建实例,而 object 则指存储在内存中的值。各种组件数据类型的含义(单独(每个数据)和聚合(对象))取决于类的设计。每个值都有特定的类型,具有特定的含义,每个值都有助于对实例的整体理解。

类设计者必须确定数据除了数据类型之外还意味着什么。例如,在Student 类中,std::string 可以保存学生的姓名。数据类型为std::string,具有其所有内置含义,但在 Student 对象的上下文中,特定值表示特定于给定学生的内容,并且可能具有特定形式。假设格式如下:姓氏、逗号、空格、名字。std::string 实例无法知道或强制执行这样的方案;因此,由类来维护它。这种方法很糟糕;最好有多个表示每条数据的字符串。这个例子表明,好的类设计可以很复杂。

std::string 的解释是学生的姓氏、一些分隔标点符号和学生的名字之一。总体而言,其他数据字段将增加 Student 实例的含义。该实例 的结果足以代表某个学术机构中的特定学生。解决问题所需的所有数据都有望在课堂上建模并在实例中表示出来。

\mySubsubsection{7.1.2.}{操作集}

操作实例数据应该仅限于类本身。如果外部代码(客户端)代码可以直接使用实例数据,则很快就会出现错误或漏洞。错误是不可避免的;因此,需要控制对数据的访问和修改。必须控制对值的含义(解释)、合法值是什么(值的范围)以及如何操作值的了解,以防止出现错误。

公共接口是客户端可以使用的一组操作(构造函数和函数)。仅将部分类成员设为公共可限制对实例的访问并确保访问正确完成。可能存在不被视为公共接口一部分的其他成员(受保护和私有的函数和析构函数)。总之,这些功能成员是一组操作。想想一个整数变量。典型的操作包括加法、减法、乘法和除法 - 所有这些操作都可以对整数进行。同样,公共接口是可以对类的实例进行的操作。

思考正确的操作集需要考虑几个用例:客户端如何使用实例并与之交互?默认情况下,类会提供一些操作,但通常这些操作不足以执行任何有意义的操作。由于类是为表示概念或实体而编写的,因此开发人员必须提供有意义的行为和与其实例交互的方式。

类设计者和开发者确定这组操作。在设计一组好的操作时,开发者应该考虑可以进行的绝对最小操作集(基础)。其他非基础操作可以用这个最小集来实现。好的设计可以减少执行计算或操作的函数数量,消除重复代码,最大限度地减少维护工作,并简化类。发现类的基础需要努力,但这是有益的。

\mySubsubsection{7.1.3.}{值范围}

值的范围与解释的概念有关。有效值通常有一端或两端有界。考虑一个 unsigned char 变量:它的低端值是 0,高端值是 255。字节的性质决定了这些界限。其他数据类型的值通常位于变量可能值的子范围内。我的机器上的整数是一个 32 位实体,其界限为正负 21.4 亿,但大多数问题不需要这个范围。例如,如果要对一个人的年龄进行建模,则值的范围需要限制为非负值,最大值可能为 150(很难说这个最大值应该是多少)。

假设 Student 类有一个名为 age 的整数变量,则可能的值范围将受到限制;使用无符号值会更好。首先,有效值应为非负值——年龄为负值是没有意义的。其次,最大值很难确定,可以任意设置为不合理 的大值 200。也许最好忽略上限范围。此范围意味着任何年龄都不应超出指定值;范围允许任何有效值并排除所有无效值。有时,范围很明显,但其他情况却很模糊。

精心设计的类必须强制每个变量可以容纳的值的范围。必须考虑每个数据字段以及所有数据字段的总体情况。每个数据字段都包含一些有关实例的重要信息,它们必须和谐地协同工作并有意义。例如,考虑一个 Movie 类,其中两个日期分别代表可供查看的第一个日期和最后一个日期。每个日期都必须保持有意义的值(不是 2 月 30 日)并正确地与另一个日期相关联。即使两个日期格式正确,最后一个查看日期早于第一个查看日期也是不正确的。

\mySubsubsection{7.1.4.}{内存机制}

如果使用动态资源,内存的分配和使用对许多类来说都是一个问题。通常,实例变量的布局遵循开发人员的偏好,可能按数据类型或字母顺序排列。然而,C++ 允许开发人员根据需要确定内存的分配位置和方式。指定具有自定义行为的分配器的能力是一种强大的一个非平凡的工具。内存布局取决于数据类型的大小和对齐方式。某些类型将落在 4 字节的起始边界上,而其他类型则落在 2 字节或 8 字节的边界上。如果实例变量定义不当,某些变量将存在未使用的内存间隙。在当今的大内存机器中,这很少会成为空间考虑的问题,但可能会成为机器性能的问题。根据对象的大小,可能需要两条或更多条缓存线,并且移动效率低下,浪费了额外的字节。理想情况下,应将很少或根本不应将额外的字节引入对象中。

\mySubsubsection{7.1.5.}{性能影响}

虽然性能不是数据类型的特征,但性能受数据类型设计和实现的影响。性能优化应该是开发程序的最后步骤之一,这并不是因为它不重要,而是因为过早的优化可能会影响正确性和有效性。设计具有良好布局的高效算法、数据结构和最小尺寸实例要好得多。性能调优应该发现瓶颈和热点并设法纠正它们。正如许多人所建议的那样,必须使用适当的工具而不是直觉来实现性能。必要时,C++ 提供了优化内存分配和对象布局的机制。如果在正确的时间有节制地使用这些选项,它们可能会发挥重要作用。
