此错误会影响正确性。指针经常用于管理动态资源;删除它们对于正常运行至关重要。通过指针访问已删除的资源通常是可行的,但行为尚未定义。

\mySamllsection{问题}

开发人员面临着许多通过原始指针管理动态资源的情况。许多现有代码都充满了这些情况。现代 C++ 使用智能指针解决了这个问题。在没有智能指针的情况下,开发人员必须小心管理动态资源。资源管理通常可以正确完成,并且不会遇到未定义的行为。然而,存在几种情况,在删除资源后访问资源,因为代码很复杂,删除需要显而易见,但事实并非如此。当从异常中恢复的代码错误处理资源时,就会发生这种情况。

就我个人而言,我曾经参与的一个项目就出现了 12 次此问题。静态代码分析器发现了这些情况,但开发人员或维护人员却没有发现。以下代码是一个非常简化的案例,说明了这个问题。

\filename{清单12.5 指针删除后访问的动态资源}

\begin{cpp}
class Person {
private:
  std::string name;
  int age;
public:
  Person(const std::string& name, int age) : name(name), age(age) {}
  const std::string& getName() { return name; }
  int getAge() { return age; }
};

int main() {
  Person* anne = new Person("Annette", 28);
  if (anne) // 1
    std::cout << anne->getName() << " is " << anne->getAge()
      << " years old\n";
  delete anne;
  if (anne) // 2
    std::cout << anne->getName() << " is " << anne->getAge()
      << " years old\n";
  return 0;
}
\end{cpp}

{\footnotesize
注释1:一个有效的测试,确保对象已被创建

注释2:一个看似有效的测试;对象已被销毁且可能被访问。
}

\mySamllsection{分析}

创建动态资源,测试其有效性,处理并删除。问题出现在删除之后,其中测试指针的有效性并错误地用于访问资源。不同的编译器和系统的组合对此代码的反应不同,但在每种情况下,任何通过已删除指针访问资源的尝试都需要纠正。在最好的情况下,程序崩溃,阻止进一步访问。这种未定义行为的情况可能看起来有效,但却是危险的错误。

\mySamllsection{解决}

解决这个问题最简单的方法是删除指针后将其置空。如果删除操作是在作用域的末尾,则可以忽略此建议,但一般情况下最好这样做。通过空指针访问比成功访问删除了资源。空指针访问导致的崩溃是编程问题的大胆陈述。以下代码显示了这一微小的更改,它产生了显著的效果。

\filename{清单12.6 指针为空以防止删除后的访问}

\begin{cpp}
class Person {
private:
  std::string name;
  int age;
public:
  Person(const std::string& name, int age) : name(name), age(age) {}
  const std::string& getName() { return name; }
  int getAge() { return age; }
};
int main() {
  Person* anne = new Person("Annette", 28);
  if (anne)
    std::cout << anne->getName() << " is " << anne->getAge() << " years old\n";
  delete anne;
  anne = NULL; // 1
  if (anne) // 2
    std::cout << anne->getName() << " is " << anne->getAge() << " years old\n";
  return 0;
}
\end{cpp}

{\footnotesize
注释1:将已删除的指针清零

注释2:阻止通过已删除的指针进行访问的测试
}

\mySamllsection{建议}

\begin{itemize}
\item
始终将已删除资源的指针清空;在经典 C++ 中使用 0,在现代C++ 中使用 nullptr。
\end{itemize}










